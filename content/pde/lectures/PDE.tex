\documentclass[12pt, oneside]{book}
\usepackage[Bjarne]{fncychap}
\usepackage{amsthm}
\usepackage{graphicx} 
\usepackage{tikz}
\usepackage{titling}
\usepackage{tikz-cd}
\usepackage[english]{babel}
\usepackage[utf8]{inputenc}
\usepackage[many]{tcolorbox}
\usepackage[a4paper, total={6.5in,10in}]{geometry}
\usepackage{amsfonts,amsmath, amssymb}
\usepackage{mathrsfs}
\usepackage{tkz-euclide}
\usepackage{pstricks}
\usepackage{tikz}
\usepackage{hyperref}
\usepackage{float}
\usepackage{lscape}
\usepackage[scaled=0.92]{helvet}
\usepackage[lite]{mtpro2}
\usepackage{bm}
\usepackage{setspace}
\usepackage{enumitem}
\usepackage{textcomp}
\usepackage{mathtools}

\hypersetup{colorlinks=true,linkcolor=blue,filecolor=magenta,urlcolor=cyan}

\usetikzlibrary{decorations.fractals, arrows.meta, quotes}
\renewcommand{\rmdefault}{ptm}
\pdfmapfile{+mtpro2.map}

\tcbuselibrary{theorems}

\tcbset{
	colback=white,
	colframe=black,
	fonttitle=\bfseries,
	boxrule=0.4pt,
	arc=0pt,
	left=6pt,
	right=6pt,
	top=6pt,
	bottom=6pt
}

\newtcbtheorem{exercise}{Exercise}
{colframe=black, fonttitle=\bfseries}{ex} \newtheorem{definition}{Definition}[section] \newtheorem{example}{Example}[section]
\newtheorem{remark}{Remark}[section]
\newtheorem{prop}{Proposition}[section]
\newtheorem{soln}{Solution}[section]
\newtheorem{lemma}{Lemma}[section]
\newtheorem{coro}{Corollary}[section]
\newtheorem{theo}{Theorem}[section]
\newtheorem{claim}{Claim}[section]
\newtheorem*{theo*}{Theorem}
\DeclarePairedDelimiter\floor{\lfloor}{\rfloor}
\onehalfspacing

\newcommand{\norm}[1]{\left\lVert#1\right\rVert}
\newcommand{\A}{\mathscr{A}}
\newcommand{\B}{\mathscr{B}}
\renewcommand{\L}{\mathscr{L}}
	
	\date{}
	\pagestyle{plain}
	\AtBeginDocument{\renewcommand{\bibname}{References}}
	\let\cleardoublepage=\clearpage
	\begin{document}

		
	\begin{titlepage}
		\centering
		
		{\Large \textbf{PARTIAL DIFFERENTIAL EQUATIONS}\par}
		\vspace{0.5cm}
		\hrulefill\par
		\vspace{0.5cm}
		{ \large International Mathematics Masters\\
			National Higher School of Mathematics (NHSM)\par}
		
		\vfill
	\hrulefill\par
		\begin{minipage}{0.35\textwidth}
			\centering
			\includegraphics[height=0.425\linewidth]{immm logo.png}
		\end{minipage}
		\hfill
		\begin{minipage}{0.35\textwidth}
			\centering
			\includegraphics[height=0.8\linewidth]{national hsm logo.png}
		\end{minipage}
		
		\vspace{1.5cm}
		
		{\large 2026\par}
		
	\end{titlepage}	
		
	\cleardoublepage   
	\tableofcontents
	
	\cleardoublepage
	\chapter{Theory of (Unbounded) Operators}	
	\section{Preliminaries on Operators}
Let $(X,\|\cdot\|)$ be a Banach space and $\mathscr{L}(X)$ be the Banach space of bounded linear operators.
\begin{definition}
An operator $\mathscr{A}:D(\mathscr{A})\longrightarrow X$ is called an \textit{unbounded linear operator (UBLO)} if $D(\mathscr{A})$ is a subspace of $X$ and $\sup_{x\neq 0}\frac{\|\mathscr{A}x\|}{\|x\|}=+\infty$
\end{definition}
\begin{exercise}{}{}
	Let $\mathscr{A}:H^{1}\longrightarrow L^2$, such that $f\mapsto f'$ and $D(\mathscr{A})=\{f:\mathbb{R}\rightarrow \mathbb{R}:f,f'\in L^2\}$. Show $\mathscr{A}$ is an unbounded linear operator.
\end{exercise}
\noindent \textbf{Notation:} If $\mathscr{A}$ and $\mathscr{B}$ are unbounded linear operators, then $\mathscr{A}\supset \mathscr{B}$ if and only if $D(\mathscr{A})\supset D(\mathscr{B})$ and for all $x\in D(\mathscr{B}), \mathscr{A}x=\mathscr{B}x$.	
\subsection{Resolvent Operator}		
\begin{definition}
	Let $\mathscr{A}:D(\mathscr{A})\longrightarrow X$ be a UBLO.
	\begin{align*}
		&\rho(\mathscr{A})=\ \text{Resolvent of $\mathscr{A}$} = \left\{\lambda\in \mathbb{C}\ \bigg| \begin{aligned}
			&(\lambda I - \mathscr{A}) : D(\mathscr{A}) \to X \text{ is bijective, and} \\
			&(\lambda I - \mathscr{A})^{-1} \in \mathscr{L}(X)
		\end{aligned}\right\}\\
		&\sigma(\mathscr{A})= \ \text{Spectrum of $\mathscr{A}$}= \mathbb{C}\setminus\rho(\mathscr{A}).
	\end{align*}
\end{definition}		
	\begin{definition}
		$\mathscr{A}$ is closed if and only if the graph of $\mathscr{A}$, denoted as $G(\mathscr{A})$ is closed. Also $\mathscr{A}$ is closable if and only if there exists $\tilde{\mathscr{A}}\supset\mathscr{A}$ such that $G(\tilde{\mathscr{A}})=\overline{G(\mathscr{A})}$.
	\end{definition}	
	\vspace{-2mm}
Note that $G(\mathscr{A})=\left\{(x,\mathscr{A}x)\big|x\in D(\mathscr{A})\right\}$.		
\begin{exercise}{}{}
	\begin{itemize}
		\item [1.] Prove that if it exists, $\tilde{\mathscr{A}}$ is unique, it then denoted by $\overline{\mathscr{A}}$ called closure of $\mathscr{A}$.
		\item [2.] Let $\mathscr{A}_{\ell}=\frac{d}{dx}$ with $(X=C^0([a,b],\mathbb{R}),\norm{\cdot}_{\infty}=\sup|f(x)|$ and $D(\mathscr{A}_{\ell})=C^{\ell}([a,b],\mathbb{R})$. Prove $\overline{\mathscr{A}_{\ell}}=\mathscr{A}_1$.
	\end{itemize}
\end{exercise}		
\begin{lemma}
	If $\mathscr{A}$ an unbounded linear operator is closable, then $\rho(\overline{\mathscr{A}})=\rho(\mathscr{A})$. If $\mathscr{A}$ is closed then $\rho(\mathscr{A})=\{\lambda\in \mathbb{C}\ | \ (\lambda I-\mathscr{A})^{-1}\in \mathscr{L}(X)\}$.\\
	\textcolor{blue}{Hints (Exercise): If $\rho(\mathscr{A})\neq \emptyset$ implies $\mathscr{A}$ is closed. (Show first that if $T$ is a UBLO with $T^{-1}\in \mathscr{L}(X)$ implies $T$ is closed)}.
\end{lemma}		
		
	\begin{coro}
		Let $\mathscr{A}:D(\mathscr{A})\subset X\longrightarrow X$ be a closed UBLO then $\sigma(\mathscr{A})=\sigma_p(\mathscr{A})\cup \sigma_r(\mathscr{A})\cup\sigma_c(\mathscr{A})$ where 
		\begin{itemize}
			\item [1.] $\sigma_p(\mathscr{A})=\left\{\lambda\in\mathbb{C}:\ker(\lambda I-\mathscr{A})\neq \{0\}\right\}$ (punctual spectrum and $\lambda$'s are the eigenvalue).
			\item [2.] $\sigma_c(\mathscr{A})=\left\{\lambda\in\mathbb{C}:\ker(\lambda I-\mathscr{A})= \{0\},\overline{\text{Rg}(\lambda I-\mathscr{A})}\subsetneq X\right\}$ (continuous spectrum).
			\item [3.] $\sigma_r(\mathscr{A})=\left\{\lambda\in\mathbb{C}:\lambda I-\mathscr{A} \ \text{is injective},\overline{\text{Rg}(\lambda I-\mathscr{A})}= X, \ \text{Rg}(\lambda I-\mathscr{A})\neq X\right\}$ (residual spectrum).
		\end{itemize}
	\end{coro}
\begin{exercise}{}{}
	Let $$X=\ell^2(\mathbb{C})=\left\{(x_n)_{n\geq 0}:\sum_n |x_n|^2<\infty\right\},$$
with $(\mathscr{A}x_n)_{n\geq 0}=\left(\frac{x_n}{1+n}\right)_{n\geq 0}$. Prove that $\mathscr{A}$ is a BLO, injective, $\overline{\text{Rg}(\mathscr{A})}=X$ and $\text{Rg}(\mathscr{A})\subsetneq X$.
\end{exercise}	
	\begin{theo}
	If $\mathscr{A}$ is a closed UBLO then $\rho(\mathscr{A})$ is open. If $\mu\in \rho(\mathscr{A})$, then for all $\lambda\in \mathbb{C}$ with $r:=|\mu-\lambda|, \|(\mu I-\mathscr{A})^{-1}\|<1$ then $\lambda\in \rho(\mathscr{A})$ and
	\vspace{-2mm}
	$$(\lambda I-\mathscr{A})^{-1}=\sum_{n=0}^{\infty}(\mu-\lambda)^n(\mu I-\mathscr{A})^{-(n+1)}$$
	\textcolor{blue}{To do Question: do you need $\mathscr{A}$ closed?}
	\end{theo}	
\begin{theo}[Resolvent Identity]
	Let $\mathscr{A}$ be a UBLO.
	For $\lambda \in \rho(A)$, define the resolvent \\ operator
	\vspace{-2mm}
	\[
	R(\lambda) := (\lambda I - A)^{-1}.
	\]
	Then for all $\lambda, \mu \in \rho(A)$,
	\[
	R(\lambda) - R(\mu) = (\mu - \lambda) R(\mu) R(\lambda).
	\]
\end{theo}	
\begin{coro}
	The mapping
	$
	\lambda \mapsto R(\lambda)
	$
	from $\rho(\mathscr{A})$ into $\mathscr{L}(X)$ is analytic. Moreover,
	\[
	\frac{d^n}{d\lambda^n} (\lambda I - A)^{-1}
	= (-1)^n n! [(\lambda I - A)^{-1}]^{(n+1)}.
	\]
\end{coro}
\subsection{Dual Operators}	
Let $X\cong X^*$ and $\mathscr{A}$ a closed UBLO with $\overline{D(\mathscr{A})}=X$ a dense UBLO.\\
If $X$ and $Y$ are Banach spaces with duals $X^*$ and $Y^*$, then for $x\in X$ and $x^*\in X^*$, we define the duality product as $\langle x^*,x\rangle$.
\begin{definition}[Dual Operator of $\mathscr{A}$]
Let
$
\mathscr{A} : D(\mathscr{A}) \subset X \to Y
$ (UBLO), be such that \\ $\overline{D(\mathscr{A})}=X$.
The \emph{dual operator}
$
\mathscr{A}^* : D(\mathscr{A}^*) \subset Y^* \to X^*
$ is a UBLO
 defined as follows:
 \vspace{-3mm}
\[
D(\mathscr{A}^*) :=
\left\{ y^* \in Y^* \,\middle|\, 
\exists z^* \in X^* \text{ such that }
\langle y^*, Ax \rangle = \langle z^*, x \rangle
\ \forall x \in D(\mathscr{A})
\right\}.
\]
and $y^* \in D(\mathscr{A}^*)$, the element $z^*$ is unique and we define
$
A^* y^* := z^*.
$	
\end{definition}		
\begin{lemma}
	Let $X, Y$ be Banach spaces and let $\mathscr{A} \in \mathscr{L}(X,Y)$.
	Then $\mathscr{A}^* \in \mathscr{L}(Y^*, X^*)$ and
	\vspace{-2mm}
	\[
	\|\mathscr{A}^*\|_{\mathscr{L}(Y^*,X^*)} = \|\mathscr{A}\|_{\mathscr{L}(X,Y)}.
	\]
\end{lemma}		
\begin{lemma}
	Let $X$ be a reflexive Banach space with $X=X^*$ and let
	$
	\mathscr{A} : D(\mathscr{A}) \subset X \to X
	$
	be a closedly dense UBLO.
	Then
	$
	\overline{D(\mathscr{A}^*)} = X^*
	(\cong X)$,
	and $\mathscr{A}^*$ is closed.
\end{lemma}		
\begin{theo}
	Let $\mathscr{A}$ be a closedly dense UBLO.
	Then
	$
	\rho(\mathscr{A}) = \rho(\mathscr{A}^*)
	$
	and for all $\lambda \in \rho(A)$,
	\vspace{-2mm}
	\[
	\bigl[(\lambda I - \mathscr{A})^{-1}\bigr]^*
	= ({\lambda} I - \mathscr{A}^*)^{-1}.
	\]
\end{theo}
\begin{exercise}{}{}
	\begin{itemize}
		\item [1.] Let $\mathscr{A}=\frac{d}{dx}$ on $X=L^2(\mathbb{R})$ and $D(\mathscr{A})=\{f\in X: f'\in L^2(\mathbb{R})\}$. Show the following:
		\begin{itemize}
			\item [a.] $\rho(\mathscr{A})=\mathbb{C}\setminus i\mathbb{R}$ which implies $\sigma(\mathscr{A})=i\mathbb{R}$.
			\item [b.] $\mathscr{A}$ is a closed unbounded linear operator.
			\item [c.] If $\lambda\in \rho(\mathscr{A})$ then $(\lambda I-\mathscr{A})^{-1}:X\longrightarrow D(\mathscr{A})$ is bounded.
		\end{itemize}
For $\Re(\lambda)\neq 0$; show for all $g\in X$, there exists uniquely $f\in D(\mathscr{A})$ such that \\ $(\lambda I-\mathscr{A})f=g$.\\
For $\Re(\lambda)= 0$; show for all $f_n\in X$ with $\|f_n\|_{\ell^2}=1$ then $(i\omega I-\mathscr{A})f_n\rightarrow 0$.
\item [2.] Do same for $\mathscr{A}=-i\frac{d}{dx}$.
	\end{itemize}
	
\end{exercise}		
		
\section{Compact Operators}
Let $X$ and $Y$ be Banach spaces on $\mathbb{K}$.

\begin{definition}
Let $K:X\to Y$ be a BLO (in $\mathscr{L}(X,Y)$), then $K$ compact iff
$
K(B_1^X(0))$ \text{ is relatively compact in } Y (i.e. $\overline{K(B_1^X(0))}$ compact).
\vspace{-2mm}
\[
\mathcal{K}(X,Y)=\{K\in \mathscr{L}(X,Y)\mid K \text{ is compact}\}.
\]
\end{definition}

\begin{exercise}{}{}
Let $
X=C([a,b],\mathbb{C})$ and $ k\in C^0([a,b]\times[c,d],\mathbb{C})
$

Define $K\in \mathscr{L}(X)$ by
\[
(Kx)(t)=\int_a^b k(t,s)x(s)\,ds.
\]
Show $K\in\mathcal{K}(X)$.	
\end{exercise}
\begin{theo}
	$\mathcal{K}(X,Y)$ is a closed subspace of $\mathscr{L}(X,Y)$.
\end{theo}
\begin{proof} We shall show this in two steps
\begin{itemize}
\item[1.] $\mathcal{K}(X,Y)$ is a vector space (Do it).
\item[2.] Closed: $K_n\to K$ (\textcolor{blue}{Prof. Yacine said he would send a different proof}).
\end{itemize}	
\end{proof}	
\begin{exercise}{}{}
Let $X=\ell^2(\mathbb{C})$,
	\[
	\mathscr{A}((x_n)_{n\ge0})=\left(\frac{x_n}{n+1}\right)_{n\ge0}.
	\]
	
	Show that $\mathscr{A}$ is compact.
\end{exercise}		
\begin{theo}
Let	$X,Y$ and $Z$ be Banach spaces on $\mathbb{K}$.
	
	\[
	X \xlongrightarrow{\mathscr{A}} Y \xlongrightarrow{\mathscr{B}} Z,
	\quad \mathscr{A}\in\mathscr{L}(X,Y),\ \mathscr{B}\in\mathscr{L}(Y,Z).
	\]
	\begin{itemize}
		\item [1.] If $\mathscr{A}$ is compact or $\mathscr{B}$ is compact, then $\mathscr{B}\mathscr{A}$ is compact.
		\item [2.] If $\mathscr{A}$ is compact then $\mathscr{A}^*\in\mathcal{K}(Y^*,X^*)$.
		\item [3.] If $\mathscr{A}$ is compact and $\operatorname{Rg}(\mathscr{A})$ is closed (in $Y$), then it is finite dimensional.
	\end{itemize}

\end{theo}		
To proceed with further results on compact operators, we need the following lemma
\begin{lemma}[Riesz Lemma]
	Let $E$ be a normed vector space, $F=\overline{F}\subset E$.
Then $\forall r\in(0,1)$, $\exists \ x_r\in E$, such that 
\vspace{-2mm}
	\[
	\|x_r\|=1,\quad d(x_r,F)\ge r.
	\]
\end{lemma}		
\begin{proof}
Since	$F\neq E$ then this implies $ \exists\ z\in E\setminus F$. Let $d=d(z,F)>0$.
	
	For $0<r<1$, $\exists\ y_r\in F$ s.t.
	\[
	0<d\le \|z-y_r\|<\frac dr.
	\]
	
	Normalize:
	\[
	x_r=\frac{z-y_r}{\|z-y_r\|},\quad \|x_r\|=1.
	\]
	
	For all $y\in F$,
	\vspace{-3mm}
	\[
	\|x_r-y\|
	=\frac{1}{\|z-y_r\|}\|z-(y_r+\|z-y_r\|y)\|
	\ge\frac d{\|z-y_r\|}>r.
	\]
\end{proof}		
\begin{prop}
Let	$\mathscr{A}\in\mathcal{K}(X)$, such that $X$ is a Banach space on $\mathbb{C}$.
If $\lambda\in\mathbb{C}^*$, then
	$
	\ker((\lambda I-\mathscr{A})^n)
	$
	has finite dimension.	
\end{prop}		
\begin{proof}[Proof. Only for $n=1$] (\textcolor{blue}{do it for $n\geq 2$}). Now, let
\vspace{-4mm}
\begin{align*}
\tilde K:=\ker(\lambda I-\mathscr{A})
=\{x\in X:\mathscr{A}x=\lambda x\}
=\left\{x\in X:x=\frac1\lambda \mathscr{A}x\right\}\subset \operatorname{Rg}(\mathscr{A}).
\end{align*}
So $\tilde K$ is closed in $\operatorname{Rg}(\mathscr{A})$.
Suppose $\dim\tilde K=+\infty$.
By Riesz lemma, $\exists\  (x_n)$ in $\tilde K$, such that
\[
\|x_n\|=1,\quad \|x_n-x_m\|\ge\frac12.
\]
Thus,
\[
\frac1{|\lambda|}\|\mathscr{A}x_n-\mathscr{A}x_m\|\ge\frac12, \qquad \forall \ n\neq m
\]
and so we have $\|\mathscr{A}x_n\|\leq \|\mathscr{A}\|$.
So $(\mathscr{A}x_n)$ is not Cauchy, hence a contradiction.

\end{proof}		
		
\begin{exercise}{}{}
	Let $X$ be a Banach space on $\mathbb{K}$. If $\mathscr{A}\in\mathscr{L}(X)$,
 assume $\exists\  n_0$ s.t.
$
\ker(\mathscr{A}^{n_0})=\ker(\mathscr{A}^{n_0+1}).
$
Then $\forall n\ge n_0$,
\[
\ker(\mathscr{A}^n)=\ker(\mathscr{A}^{n_0}).
\]
\end{exercise}		

\begin{prop}\label{1.2.2}
Let	$\mathscr{A}\in\mathcal{K}(X)$ and $X$ be a Banach space on $\mathbb{C}$, $\lambda\neq0$.
	Then $\exists \ n_0$ such that
	\[
	\forall n\ge n_0,\quad
	\ker((\lambda I-\mathscr{A})^n)=\ker((\lambda I-\mathscr{A})^{n_0}).
	\]
\end{prop}		
\begin{proof}
	Using the previous exercise and arguing by contradiction, that for all $n\geq 1$ $\ker((\lambda I-\mathscr{A})^n)\subset \ker((\lambda I-\mathscr{A})^{n+1})$ and each of them is closed.\\
	\textbf{RL:} with $r=\frac{1}{2}$ with $(x_n)_{n\geq 1} \in X$, such that $\|x_n\|=1$. Then, $x_n\in \ker((\lambda I-\mathscr{A})^{n+1})$. Thus,
	$$d(x_n,\ker((\lambda I-\mathscr{A})^n)))\geq \frac{1}{2}.$$
For $n=1,x\in \ker(\lambda I-\mathscr{A})\ \Rightarrow\ x=\frac{\mathscr{A}}{\lambda}x.$
For all $1\le m<n$,
\begin{align*}
	\frac{\mathscr{A}x_n}{\lambda}-\frac{\mathscr{A}x_m}{\lambda}
	&= x_n-x_n+\frac{\mathscr{A}x_n}{\lambda}-\left(x_m-x_m-\frac{\mathscr{A}x_m}{\lambda}\right) \\
	&= x_n-\left[\frac{(\lambda I-\mathscr{A})x_n}{\lambda}+x_m-\frac{(\lambda I-\mathscr{A})x_m}{\lambda}\right].
\end{align*}
So,

\[
\left\|\frac{\mathscr{A}x_n}{\lambda}-\frac{\mathscr{A}x_m}{\lambda}\right\|
\ge d\bigl(x_n,\ker(\lambda I-\mathscr{A})^n\bigr)
\ge \frac12.
\]

\medskip
which is a contradiction.
\end{proof}		
Notice that if
$
\ker(\lambda I-\mathscr{A})\ne \{0\},
$
then
$
\lambda\in \sigma_p(\mathscr{A}).
$ Notice,
\vspace{-2mm}
\[
\dim \ker(\lambda I-\mathscr{A})=\text{geometric multiplicity}.
\]

With Proposition \ref{1.2.2} $\Rightarrow$ $\exists\, n_0$ (smallest one) such that
\vspace{-2mm}
\[
\ker\bigl((\lambda I-\mathscr{A})^{n_0}\bigr)
=
\ker\bigl((\lambda I-\mathscr{A})^n\bigr),
\qquad \forall n\ge n_0.
\]
Note that,
\[
\ker\bigl((\lambda I-\mathscr{A})^{n_0}\bigr)
:=\text{generalized eigenspace}.
\]

\[
\dim\ker\bigl((\lambda I-\mathscr{A})^{n_0}\bigr)
:=\text{algebraic multiplicity of }\lambda.
\]
\hspace{3mm}		
\begin{prop}[Fredholm aternative]
Let	$\mathscr{A}\in\mathcal{K}(X)$, and let $X$ be a Banach space on $\mathbb{C}$.
\vspace{-2mm}	
	\[
	\operatorname{Rg}(\lambda I-\mathscr{A})=X
	\iff
	\ker(\lambda I-\mathscr{A})=\{0\}.
	\]
\end{prop}		
\begin{prop}
Let $\mathscr{A}\in\mathcal{K}(X)$, and let $X$ be a Banach space on $\mathbb{C}$, $\dim X=\infty$.
If $\lambda_n\to\lambda$,
$\lambda_n\in\sigma(\mathscr{A})\setminus\{0\}$,
pairwise distinct,then $\lambda=0$.
Hence every $\lambda\in\sigma(\mathscr{A})\setminus\{0\}$ is isolated.
\end{prop}
\begin{proof}
Let $\lambda_n \in \sigma_p(\mathscr{A})$, $\exists \ \|x_n\|=1$ such that
$
\mathscr{A}x_n=\lambda_n x_n.
$ Let
\[
X_n=\mathrm{span}(x_1,\ldots,x_n), \qquad X_n \subset X_{n+1}.
\]

Let us prove that $\dim X_n=n$.

\medskip

By induction: $n=1$ is OK.

\[
\dim X_n=n \ \Rightarrow\ \dim X_{n+1}=n+1.
\]
By contradiction, $x_{n+1}\in X_n$.

\[
x_{n+1}=\sum_{i=1}^n \alpha_i x_i,\
\text{which implies} \quad
\lambda_{n+1} x_{n+1}
=
\sum_{i=1}^n \alpha_i \lambda_{n+1} x_i.
\]
Thus,
\[
\mathscr{A}x_{n+1}
=
\sum_{i=1}^n \alpha_i \lambda_i x_i.
\]

Hence,
\[
0=\sum_{i=1}^n \alpha_i(\lambda_{n+1}-\lambda_i)x_i.
\]

Since $(x_i)$ are linearly independent,
\[
\alpha_i(\lambda_{n+1}-\lambda_i)=0,
\qquad 1\le i\le n.
\]

which implies
$\Rightarrow \alpha_i=0.$
$
\Rightarrow x_{n+1}=0, (\text{Impossible}).
$

\medskip

Notice:
\[
(\lambda_n I-\mathscr{A})X_n \subset X_{n-1},\qquad \forall n\ge2.
\]

\medskip

\textbf{Recall:}
\[
\|y_n\|=1,\qquad y_n\in X_n,
\]
\[
d(y_n,X_{n-1})\ge \frac12.
\]

\medskip

For $2\le m<n$,
\begin{align*}
	\left\|\frac{\mathscr{A}y_n}{\lambda_n}-\frac{\mathscr{A}y_m}{\lambda_m}\right\|
	&=
	\left\|y_n-
	\left[
	\frac{\lambda_n I-\mathscr{A}}{\lambda_n}y_n
	+y_m+
	\frac{\lambda_m I-\mathscr{A}}{\lambda_m}y_m
	\right]\right\| \\
	&\ge d(y_n,X_{n-1})
	\ge \frac12.
\end{align*}

Assume that
\[
\lambda_n \to \lambda \quad (n\to\infty).
\]

Suppose $\lambda\ne0$, then
\[
\left|\frac1{\lambda_n}\right|\le C_0
\quad \text{for $n$ large enough}.
\]

Then
\[
\left(\frac{\mathscr{A}y_n}{\lambda_n}\right)_{n\ge1}
\]
is a bounded sequence.

Then we have built a sequence in
$
\mathscr{A}(B_M^X(0))
$, $M>0$
which does not admit a convergent subsequence. Which is a Contradiction.
\end{proof}

\begin{theo}
Let	$\mathscr{A}\in\mathcal{K}(X)$, and let $X$ be a Banach space on $\mathbb{C}$. Then $\sigma(\mathscr{A})\setminus\{0\}$ is made of eigenvalues, contains a countable number of points and the set of accumulation points contained in $\{0\}$.
\end{theo}

\subsection*{Main use of compact operators (in PDEs)}
They appear as ``inverse'' of UBLO.

\begin{definition}
Let $\mathscr{A}:D(\mathscr{A})\subset X\to X$ UBLO, closed, $\rho(\mathscr{A})\neq\{0\}$.
$\mathscr{A}$ is said to have compact resolvent if
\vspace{-2mm}
\[
(\lambda I-\mathscr{A})^{-1}\in\mathcal{K}(X),
\quad \forall\lambda\in\rho(\mathscr{A}).
\]
\end{definition}
\textbf{Main Example:}
$\mathscr{A}=-\Delta$ on $\Omega$ with
$\mathscr{A}u=-u_{xx}$.

\section{Adjoints, Symmetric and Self-adjoint Operators}
Let $\mathscr{H}$ be a Hilbert space, with inner product
\vspace{-2mm}
\[
\langle\cdot,\cdot\rangle_{\mathscr{H}}:\mathscr{H}\times \mathscr{H}\to\mathbb{C}.
\]

$\mathscr{A}:D(\mathscr{A})\subset \mathscr{H}\to \mathscr{H}$ UBLO, $\overline{D(\mathscr{A})}=\mathscr{H}$.
\begin{definition}[Adjoint Operator $\mathscr{A}^{\circ}$]
\vspace{-2mm}
\[
D(\mathscr{A}^{\circ})=\left\{x\in \mathscr{H}:v\mapsto \langle \mathscr{A}v,x\rangle_{\mathscr{H}}: D(\mathscr{A})\rightarrow \mathbb{C}\ \text{bdd operator}\right\}.
\]
If $x\in D(\mathscr{A}^{\circ})$, then there exists uniquely $z\in \mathscr{H}$ such that $\langle v,z\rangle_{\mathscr{H}}=\langle\mathscr{A}v,x\rangle_{\mathscr{H}}$ for all \\$v\in D(\mathscr{A})$.\\
Observe, with Riesz representation and the fact that $\overline{D(\mathscr{A})}=\mathscr{H}$, we have that $z:=\mathscr{A}^{\circ}x$ and $\langle v,\mathscr{A}^{\circ}x\rangle_{\mathscr{H}}=\langle \mathscr{A}v,x\rangle_{\mathscr{H}}$ for all $v\in D(\mathscr{A})$.
\end{definition}
\begin{remark}
Let $\mathscr{H}$ be a Hilbert space,
$E:\mathscr{H}\to \mathscr{H}^*$,
$
x\mapsto\langle x,\cdot\rangle.
$
Linear isometry between $\mathscr{H}$ and $\mathscr{H}^*$. (One can identify $\mathscr{H}$ and $\mathscr{H}^*$). Now, we define the Dual operator as the following:
\vspace{-2mm}
\[
\mathscr{A}^*:D(\mathscr{A}^*)\subset \mathscr{H}^*\to \mathscr{H},\quad
\mathscr{A}^{\circ}=E^{-1}\mathscr{A}^*E.
\]	
\end{remark}
\begin{definition}[Symmetric and Self-adjoint Operator]
Let $(\mathscr{H},\langle,\rangle)$ be a Hilbert space and $\mathscr{A}: D(\mathscr{A})\subset \mathscr{H}\rightarrow \mathscr{H}$ be a UBLO, then
\begin{itemize}
	\item [1.] \textbf{Symmetric:} If $\overline{D(\mathscr{A})}=\mathscr{H}$ and $\mathscr{A}^{\circ}\supset \mathscr{A}$ with $D(\mathscr{A}^{\circ})\supset D(\mathscr{A})$ and for all $x,y\in D(\mathscr{A}),$ $\langle \mathscr{A}x,y\rangle=\langle x,\mathscr{A}y\rangle$.
	\item [2.] \textbf{Self-adjoint:} If $\overline{D(\mathscr{A})}=\mathscr{H}$ and $\mathscr{A}^{\circ}= \mathscr{A}$.
\end{itemize}	
\end{definition}
\begin{exercise}{}{}
\begin{itemize}
	\item [1.]Let $\mathscr{H}$ be a Hilbert space, $\mathscr{A}:D(\mathscr{A})\subset \mathscr{H}\to \mathscr{H}$, $\overline{D(\mathscr{A})}=\mathscr{H}$. If $\mathscr{A}$ is closed, then $\overline{D(\mathscr{A}^{\circ})}=\mathscr{H}$.
	\item [2.] Let $\mathscr{H}$ be a Hilbert space, $\mathscr{A}:D(\mathscr{A})\subset \mathscr{H}\to \mathscr{H}$, $\overline{D(\mathscr{A})}=\mathscr{H}$. Suppose $\mathscr{A}$ is symmetric and if $\lambda \in \sigma_{p}(\mathscr{A})$, then prove that $\lambda\in\mathbb{R}$ and
	\[
	\inf_{\|x\|=1,x\in D(\mathscr{A})}\langle \mathscr{A}x,x\rangle
	\le \lambda \le
	\sup_{\|x\|=1,x\in D(\mathscr{A})}\langle \mathscr{A}x,x\rangle.
	\]
\end{itemize}
\end{exercise}
\begin{prop}
Let	$(\mathscr{H},\langle\cdot,\cdot\rangle)$ be an Hilbert space over $\mathbb{C}$, If $\mathscr{A}$ is self-adjoint, injective and $\overline{D(\mathscr{A})}=\mathscr{H}$. 	
Then $\mathscr{A}^{-1}:\mathrm{Rg}(\mathscr{A})\to \mathscr{H}$ is self-adjoint.
\end{prop}
\begin{proof}
Since $\mathscr{A}$ is injective then $\mathscr{A}^{-1}$ is well defined, and since $\mathscr{A} = \mathscr{A}^{\circ}$ then $\mathscr{A}$ is closed.

Now assume $(x_n) \subset D(\mathscr{A})$ such that $x_n \to x \in D(A)$ (because $\mathscr{A}$ is closed) and $\mathscr{A}x_n -\to y$ then for all $z\in D(\mathscr{A})$, $\langle\mathscr{A} x_n,z\rangle=\langle x_n,\mathscr{A}^{\circ}z\rangle$ which implies $\langle y,z\rangle=\langle x,\mathscr{A}^{\circ}z\rangle$ and so we have $x\in D(\mathscr{A}^{\circ})=D(\mathscr{A})$ and $y=\mathscr{A}^{\circ}$. Notice $\overline{\mathrm{Rg}(\mathscr{A})}=\ker(\mathscr{A})^{\perp}$ (because of self-adjointness).\\
Injectivity implies $\overline{\mathrm{Rg}(\mathscr{A})}=\mathscr{H}$ which implies $\overline{D(\mathscr{A}^{-1})}=\mathscr{H}$. So $\mathscr{A}^{-1}$ is densely defined.\\
Now, observe for all $u,v\in D(\mathscr{A}^{-1})$, $u=\mathscr{A}^{\circ}x$ and $v=\mathscr{A} y$ with $x,y\in D(\mathscr{A})$. Hence,
\vspace{-2mm}
$$\langle \mathscr{A}^{-1}u,v\rangle =\langle x,\mathscr{A}y\rangle=\langle \mathscr{A} x,y\rangle =\langle x, \mathscr{A}^{-1}y\rangle.$$
To this end, $
(\mathscr{A}^{-1})^{\circ} \subset \mathscr{A}^{-1},$
$
\forall\ z \in D\big((\mathscr{A}^{-1})^{\circ}\big)
$
$
\exists\  w,\forall \ u \in D(\mathscr{A}^{-1}) = R(\mathscr{A}) \; (i.e.\ u= \mathscr{A}x)
$
\[
\langle \mathscr{A}^{-1}u, z \rangle
= \langle u, w \rangle
\Rightarrow \forall x \in D(\mathscr{A})
\quad
\langle x, z \rangle
= \langle \mathscr{A}x, w \rangle
\]
By definition $w \in D(\mathscr{A}^{\circ})$ and $\mathscr{A}^{\circ} w = z$.
$
\mathscr{A}w = z
\Rightarrow
z \in Rg(\mathscr{A}) = D(\mathscr{A}^{-1}).
$
\end{proof}

\begin{theo}
	Let $\mathscr{H}$ be a Hilbert space, suppose $\mathscr{A}:D(\mathscr{A})\subset \mathscr{H}\to \mathscr{H}$ is symmetric and surjective then $\mathscr{A}$ is self-adjoint.
\end{theo}
\begin{proof}
$\mathscr{A}$ and $\mathscr{A}^{\circ}$ are injective. Do it only for $\mathscr{A}$, let $x \in D(\mathscr{A})$ and $\mathscr{A}x = 0$.
\[
\forall y \in D(\mathscr{A}),
\quad
0 = \langle \mathscr{A}x, y \rangle
= \langle x, \mathscr{A}y \rangle
\Rightarrow x \perp Rg(\mathscr{A}) = \mathscr{H}.
\]
which implies $x=0$.\\
Next, we show $\mathscr{A}$ closed. 
\[
(x_n)_{n \ge 1} \subset D(\mathscr{A}),
\quad
x_n \to x \text{ in } \mathscr{H},
\quad
\mathscr{A}x_n \to y \text{ in } \mathscr{H}
\]
We shall show $y = \mathscr{A}x$. Now,
$
\forall z \in D(\mathscr{A})\ \text{then} \ \langle \mathscr{A}x_n, z \rangle
= \langle x_n, \mathscr{A}z \rangle
$, which implies
$
\langle y, z \rangle = \langle x, \mathscr{A}z \rangle
$ which implies $x \in D(\mathscr{A}^{\circ}) \text{ and } y = \mathscr{A}^{\circ}x$. Since
$\mathscr{A}$ surjective $\Rightarrow \exists w \in D(\mathscr{A})$
s.t.\ $\mathscr{A}w = y$ and $\mathscr{A}^{\circ}x = y$.\\
Since $\mathscr{A}$ is symmetric: $\mathscr{A}^{\circ}w = \mathscr{A}w$.
Then $\mathscr{A}^{\circ}w = \mathscr{A}^{\circ}x$, $\mathscr{A}$ is injective $\Rightarrow w = x$.
Hence $\mathscr{A}x = \mathscr{A}w = y \Rightarrow y = \mathscr{A}x \Rightarrow \mathscr{A}$ is closed.\\
By closed graph theorem both $\mathscr{A}$ and $\mathscr{A}^{-1} \in \mathscr{L}(X)$. We can conclude that $\mathscr{A}$ is a self-adjoint operator.
\end{proof}
\begin{exercise}{}{}
	Let $
	\mathscr{H} = L^2(0,\pi)$ with $
	\mathscr{A} : D(\mathscr{A}) \subset \mathscr{H} \to \mathscr{H}$ and
	\vspace{-2mm}
	\[
	\mathscr{A}f = -f''
	\]
	If $
	D(\mathscr{A}) = \{ u \in C^2 : u(0) = u(\pi) = 0 \}
	$ is $\mathscr{A}$ a self-adjoint operator?\\
Similarly, if $
	D(\mathscr{A}) = \{ u \in C^2 \mid u'(0) = u'(\pi) = 0 \}
	$ is $\mathscr{A}$ a self-adjoint operator?
\end{exercise}

\begin{theo}[Fredrich's Extension]
Let $\mathscr{H}$ be a Hilbert space on $\mathbb{C}$ with $
\mathscr{A} : D(\mathscr{A}) \subset \mathscr{H} \to \mathscr{H}$, symmetric then $\mathscr{A}$ admits a unique self adjoint extension. If either 
\begin{itemize}
	\item [a.] $\inf_{\|x\|=1,x\in D(\mathscr{A})}\langle \mathscr{A}x,x\rangle >-\infty$
	\item [b.] $\sup_{\|x\|=1,x\in
	 D(\mathscr{A})}\langle\mathscr{A}x,x\rangle<+\infty$
	\end{itemize}
such that, $\mathscr{A}\subset \mathscr{A}^{\circ}\subset (\mathscr{A}^{\circ})^{\circ}\subset \cdots$. If (a) or (b) holds, then; $\mathscr{A}\subset \mathscr{A}^{\circ}=(\mathscr{A}^{\circ})^{\circ}$.

\end{theo}

\section{Dissipative Operator and Numerical range}
\begin{definition}[Duality Map]
	Let $X$ be a Banach space on $\mathbb{K}$. The duality map is defined as $J:X\longrightarrow 2^{X^*}$, $x\mapsto J(x)=\{x^*\in X^*\mid Re\langle x^*,x\rangle=\|x\|^2, \|x^*\|_{X^*}=\|x\|_X\}$.\\
	By the Hahn-Banach theorem, $J(x)\neq \emptyset$.
\end{definition}
\textbf{Question:} What can you say about $J(X)$ when $X$ is an Hilbert space or Reflexive?
\begin{definition}
	A map $\mathscr{A}:D(\mathscr{A})\subset X\longrightarrow X$ (UBLO) is dissipative if for all $x\in D(\mathscr{A})$, there exists $x^*\in J(X)$ such that $Re\langle x^*,\mathscr{A}x\rangle \leq 0$.
\end{definition}
\begin{lemma}\label{lem1.4.1}
	$\mathscr{A}$ is dissipative if and only if for all $\lambda> 0$, $x\in D(\mathscr{A})$ we have that \\ $\|(\lambda I-\mathscr{A})x\|\geq \lambda \|x\|$.
\end{lemma}
\begin{proof}
let $x^* \in J(x)$. Then
\[
\|(\lambda I - \mathscr{A})x\| \, \|x^*\|
\ge \left| \langle x^*, (\lambda I - \mathscr{A})x \rangle \right|
\ge \Re \langle x^*, (\lambda I - \mathscr{A})x \rangle,
\]
\[
= \lambda \Re \langle x^*, x \rangle - \Re \langle x^*, \mathscr{A}x \rangle
\ge \lambda \|x\|^2.
\]

Hence, if $\|x\| \neq 0$, then
\[
\|(\lambda I - \mathscr{A})x\| \ge \lambda \|x\|.
\]

\medskip

$(\Leftarrow)$ Let $x \in D(\mathscr{A})$, $x \neq 0$, and $\lambda > 0$.
Let $y_\lambda^* \in J((\lambda I - \mathscr{A})x)$ and set
$
g_\lambda^* = \frac{y_\lambda^*}{\|y_\lambda^*\|}.
$ Then
\[
\|(\lambda I - \mathscr{A})x\|^2
= \|(\lambda I - \mathscr{A})x\| \, \|y_\lambda^*\|
= \Re \langle y_\lambda^*, (\lambda I - \mathscr{A})x \rangle.
\]

Since $y_\lambda^* \neq 0$, we have
\[
\lambda\|x\|\leq \|(\lambda I - \mathscr{A})x\|
= \Re \langle g_\lambda^*, (\lambda I - \mathscr{A})x \rangle
= \lambda \langle g_\lambda^*, x \rangle
- \Re \langle g_\lambda^*, \mathscr{A}x \rangle.
\]

Hence,
\[
\Re \langle g_\lambda^*, \mathscr{A}x \rangle
\le \lambda \langle g_\lambda^*, x \rangle - \lambda\|x\|
\le \|g_\lambda^*\| \, \|x\| = \|x\|.
\]

Therefore,
\[
\Re \langle g_\lambda^*, \mathscr{A}x \rangle \le 0. \tag{**}
\]

\medskip

\textbf{Idea:} Let $\lambda \to +\infty$.

\medskip

Unit ball in $X^*$ is compact for weak$^*$ topology
(Banach--Alaoglu).

\medskip

(Up to subsequence)
\[
g_\lambda^* \longrightarrow g^* \in X^*,
\qquad \|g^*\| \le 1.
\]

\medskip

Then from $(**)$,
\[
\Re \langle g^*, \mathscr{A}x \rangle \le 0.
\]

\medskip

$(*) \qquad
\|x\| \le \langle g_\lambda^*, x \rangle
- \frac{1}{\lambda} \Re \langle g_\lambda^*, \mathscr{A}x \rangle.
$

\medskip

Let $\lambda \to +\infty$. Then
$
\|x\| \le \langle g^*, x \rangle.
$
Hence,
$
\|g^*\| = 1
\quad \text{and} \quad
\langle g^*, x \rangle = \|x\|.
$\\
Set $x^* = \|x\| g^*$.
Then
\[
\|x^*\| = \|x\|
\quad \text{and} \quad
\langle x^*, x \rangle = \|x\|^2,
\]
that is,
$
x^* \in J(x).
$	
\end{proof}


\begin{theo}[Lumer-Phillips]\label{Lumer-Phillips}
Let $X$ be a Banach space and
$
\mathscr{A} : D(\mathscr{A}) \subset X \longrightarrow X
$
be a UBLO.
Assume that $\mathscr{A}$ is dissipative and that there exists $\lambda_0 > 0$
such that
$
\operatorname{Rg}(\lambda_0 I - \mathscr{A}) = X.
$

Then $\mathscr{A}$ is closed, $\rho(\mathscr{A}) \supset \mathbb{R}_+^*$, and
for all $\lambda > 0$,
\vspace{-3mm}
\[
\|(\lambda I - \mathscr{A})^{-1}\|_{\mathscr{L}(X)}
\le \frac{1}{\lambda}.
\tag{1.4}
\]
\end{theo}

\begin{proof} Let $\lambda_0 > 0$.

\medskip

(1) To prove: $(\lambda_0 I - A)$ is bijective.

\medskip

\begin{itemize}
	\item Surjective: Assumption,
	\item Injective: Lemma \ref{lem1.4.1}.
\end{itemize}

Hence,
\[
(\lambda_0 I - \mathscr{A})^{-1} : X \longrightarrow X
\]
is well-defined and linear.

\medskip

It is bounded: since bijective, for any $y \in X$,
there exists a unique $y \in X$ such that
\[
x = (\lambda_0 I - \mathscr{A})^{-1} y,
\qquad
(\lambda_0 I - \mathscr{A})x = y.
\]

By Lemma \ref{lem1.4.1},
\[
\frac{1}{\lambda_0} \|y\|
\ge \|(\lambda_0 I - \mathscr{A})^{-1} y\|.
\]

Hence,
\[
\|(\lambda_0 I - \mathscr{A})^{-1}\|_{\mathscr{L}(X)}
\le \frac{1}{\lambda_0},
\qquad
\lambda_0 \in \rho(\mathscr{A}).
\]

\medskip

(2) $\mathscr{A}$ is closed.

\medskip

Let $x_n \to x$, $x_n \in D(\mathscr{A})$,
and
$
Ax_n \to y.
$

Then
\[
(\lambda_0 I - \mathscr{A})x_n \to \lambda_0 x - y.
\]
Since $(\lambda_0 I - \mathscr{A})^{-1} \in \mathcal{L}(X)$, we have
\[
x_n \longrightarrow (\lambda_0 I - \mathscr{A})^{-1}(\lambda_0 x - y) = x.
\]

Hence,
\[
\lambda_0 x - y = (\lambda_0 I - \mathscr{A})x
\iff y = \mathscr{A}x.
\]

Therefore, $\mathscr{A}$ is closed.



(3) $\rho(\mathscr{A}) \supset \mathbb{R}_+^*$ and $(1.4)$.

\medskip

Since $\mathscr{A}$ is closed and $\rho(\mathscr{A}) \neq \emptyset$,
we know that $\rho(\mathscr{A})$ is open.

\medskip

Let
$
\Lambda = \rho(\mathscr{A}) \cap \mathbb{R}_+^*,
$
which is open in $\mathbb{R}_+^*$.
We show that it is closed.

\medskip

Let $(\lambda_n)_{n \in \mathbb{N}} \subset \Lambda$
such that
$
\lambda_n \longrightarrow \lambda \in \mathbb{R}_+^*.
$
Note that since $\lambda_n \in \Lambda$, we have
\[
\|(\lambda_n I - \mathscr{A})^{-1}\|_{\mathscr{L}(X)}
\le \frac{1}{\lambda_n}.
\]
We write
\[
(\lambda I - \mathscr{A})
= \bigl[I + u_n\bigr](\lambda_n I - \mathscr{A})
\Longrightarrow
\lambda I - \mathscr{A}
= \lambda_n I - \mathscr{A} + u_n(\lambda_n I - \mathscr{A}),
\]
\[
\Longleftrightarrow
(\lambda - \lambda_n)I = u_n(\lambda_n I - \mathscr{A})
\Longleftrightarrow
(\lambda - \lambda_n)(\lambda_n I - \mathscr{A})^{-1} = u_n.
\]

Hence,
\[
\|u_n\|
\le |\lambda - \lambda_n|
\|(\lambda_n I - \mathscr{A})^{-1}\|_{\mathcal{L}(X)}
\le \frac{|\lambda - \lambda_n|}{\lambda_n}.
\]

\medskip

For $n$ large enough,
\[
\frac{|\lambda - \lambda_n|}{\lambda_n} \le \frac{1}{2}.
\]

\medskip

It follows that $\lambda \in \rho(\mathscr{A})$.
Hence, $\Lambda$ is closed, and therefore
$
\Lambda = \mathbb{R}_+^*.
$

\end{proof}
\begin{coro}
	Let $X$ be a Banach space and
	$
	\mathscr{A} : D(\mathscr{A}) \subset X \longrightarrow X
	$
	be a UBLO, closed, with $\overline{D(\mathscr{A})}=X$. Assume that $\mathscr{A}$ and $\mathscr{A}^*$ are dissipative. Then
	\[
	\rho(\mathscr{A}) \supset \mathbb{R}_+^*,
	\qquad
	\forall \lambda > 0,
	\quad
	\lambda \|(\lambda I - \mathscr{A})^{-1}\| \le 1.
	\]
\end{coro}
\begin{proof}
It is easy to show that
$
\operatorname{Rg}(I - A) = X
\quad
(\text{i.e. } \lambda_0 = 1 \text{ + Theorem \ref{Lumer-Phillips}}).
$\\
$\mathscr{A}$ dissipative and closed implies
\[
\operatorname{Rg}(I - \mathscr{A})
\ \text{is a closed subspace of } X.
\]
(give details!!!)

\medskip

Let $x^* \in X^*$ such that
\[
\langle x^*, (I - \mathscr{A})x \rangle = 0,
\qquad
\forall x \in D(\mathscr{A}).
\tag{$**$}
\]

Let us prove that $x^* = 0$.

\medskip

Then $x^* \in D(\mathscr{A}^*)$ and
\[
(I - \mathscr{A}^*)x^* = 0.
\]

Since $\mathscr{A}^*$ is dissipative, by Lemma \ref{lem1.4.1}, we have 
$
x^* = 0.
$
This implies that
\[
\overline{\operatorname{Rg}(I - \mathscr{A})} = X.
\]

\medskip

Since $\operatorname{Rg}(I - \mathscr{A})$ is closed,
we obtain
\[
\operatorname{Rg}(I - \mathscr{A}) = X.
\]
\textcolor{blue}{By contradiction and using Hahn--Banach.}
\end{proof}


\begin{definition}[Numerical Range]
Let
$
\mathscr{A}: D(\mathscr{A}) \subset X \longrightarrow X
$
be UBLO. The numerical range of $\mathscr{A}$, denoted by $W(\mathscr{A})$,
\vspace{-3mm}	
\[
W(\mathscr{A})
=
\left\{
\langle x^*, \mathscr{A}x \rangle
\; \middle| \;
x^* \in J(x),\ x \in D(\mathscr{A}),
\ \|x\| = \|x^*\| = 1,
\ \langle x^*, x \rangle = 1
\right\}.
\]

\medskip

In case of a Hilbert space, we have that
$
W(\mathscr{A})
=
\left\{
\langle x, \mathscr{A}x \rangle
\; \middle| \;
x \in D(\mathscr{A}),\ \|x\| = 1
\right\}.
$

\medskip

Linear algebra in finite dimension $\mathscr{A} \in \mathcal{M}_n(\mathbb{K})$, we have that $W(\A) = \left\{ \langle x, \A x \rangle \; \middle| \; \norm{x} = 1 \right\}$.

\end{definition}

\begin{theo}[\textcolor{blue}{Home-work}]
Let
$
\mathscr{A} : D(\mathscr{A}) \longrightarrow X
$
be closed, with $\overline{D(\mathscr{A})} = X$.

\medskip

1) If $\lambda \notin \overline{W(\mathscr{A})}$, then $(\lambda I - \mathscr{A})$ is injective,
has closed image, and
for all $x \in D(\mathscr{A})$,
\[
\|(\lambda I - \mathscr{A})x\|
\ge d(\lambda, W(\mathscr{A})) \, \|x\|.
\]

\medskip

Moreover, if $\lambda \in \rho(\mathscr{A})$, then
\[
\|(\lambda I - \mathscr{A})^{-1}\|_{\mathscr{L}(X)}
\le \frac{1}{d(\lambda, W(\mathscr{A}))}.
\tag{$**$}
\]

\medskip

2) If $\Lambda$ is a connected open subset of
$\mathbb{C} \setminus W(\mathscr{A})$
such that
$
\rho(\mathscr{A}) \cap \Lambda \neq \emptyset,
$
then
$
\rho(\mathscr{A}) \supset \Lambda
$
and $(**)$ holds true.	
\end{theo}


\chapter{Introduction to the theory of Semi-groups}
\section{Intro to the Introduction}
\begin{definition}
	Let $X$ be a Banach space over $\mathbb{K}$. A one-parameter family of bounded linear operators on $X$, $(T(t))_{t \geq 0}$, is a semigroup (SG) of bounded linear operators on $X$ if:
	
	\begin{enumerate}
		\item $T(0)=Id_X$,
		\item $\forall (t,s)\in \mathbb{R}_+^2 : \; T(t+s)=T(t)\cdot T(s)$ \hfill (SG property).
	\end{enumerate}
\end{definition}
\begin{remark}
$T(t)$ and $T(s)$ commute.	
\end{remark}

2. \textbf{Infinitesimal generator of SG--LO $(T(t))_{t \geq 0}$}
	
	Let
	$\mathscr{A} : D(\mathscr{A})\subset X \longrightarrow X$ be an unbounded linear operator with
	\[
	D(\mathscr{A})=\left\{ x\in X \; \middle| \; 
	\lim_{t\to 0^+}\frac{T(t)x-x}{t} \text{ exists} \right\}.
	\]
	
	and
	\[
	\lim_{t\to 0^+}\frac{T(t)x-x}{t}=\mathscr{A}x, 
	\qquad x\in D(\mathscr{A}).
	\]
where $D(\mathscr{A})=\text{domain of } \mathscr{A}$.
	
\section{Uniformly Continuous SG--BLO}	
\begin{definition}
	A SG--BLO on $X$, $(T(t))_{t\geq 0}$ is uniformly continuous if
	\vspace{-3mm}
	 $$\|T(t)-Id\|_{{\mathscr{L}}(X)}\rightarrow 0 \quad \text{as} \ t\to 0^+$$
\end{definition}
\begin{lemma}\label{lem2.1}
Let $(T(t))_{t \geq 0}$ be a SG--BLO which is uniformly continuous.
Then, $\forall t>0$,
\vspace{-3mm}
\[
\|T(s)-T(t)\| \xrightarrow[s\to t]{} 0
\]
(continuity everywhere).
\end{lemma}
\begin{proof}
	Let $t$ be fixed.
$
	T(s)=T(s-t+t), \qquad s-t \geq 0.
$
\[
	s \geq t \;\Rightarrow\; T(s)=T(s-t)T(t)
	\Rightarrow\; T(s)-T(t)=T(t)\,[T(s-t)-I_d]
	\]
	\[
	\|T(s)-T(t)\| \leq \|T(t)\|\,\|T(s-t)-I_d\|
	\xrightarrow[s\to t]{} 0.
	\]
For	$s \leq t$
	
	\[
	T(t)=T(t-s)T(s)
\Rightarrow
	T(t)-T(s)=T(s)\,[T(t-s)-I_d]
	\]
	
$
	\text{(Prove that } \sup_{[0,t]}\|T(t)\|<+\infty \text{)}
$
	
	Then
	\[
	\|T(t)-T(s)\| \leq \|T(s)\|\,\|T(t-s)-I_d\|
	\]
	
	\[
	\leq \sup \|T(s)\|\,\|T(t-s)-I_d\|
	\xrightarrow[s\to t]{} 0.
	\]
	
\end{proof}

\begin{theo}\label{theorem 2.2.1}
	A linear operator $\mathscr{A}$ is the infinitesimal generator of a uniformly
	continuous semigroup if and only if $\mathscr{A}$ is a bounded linear operator.
\end{theo}
\begin{proof}
	Let $\mathscr{A}$ be a bounded linear operator on $X$ and set
	\[
	T(t)=e^{t\mathscr{A}}=\sum_{n=0}^{\infty}\frac{(t\mathscr{A})^n}{n!}.
	\tag{1.5}
	\]
	
	The right-hand side of (1.5) converges in norm for every $t\geq 0$ and defines,
	for each such $t$, a bounded linear operator $T(t)$. It is clear that $T(0)=I$
	and a straightforward computation with the power series shows that
	$T(t+s)=T(t)T(s)$. Estimating the power series yields
	\[
	\|T(t)-I\|\leq |t|\|\mathscr{A}\|e^{|t|\|\mathscr{A}\|}
	\]
	and
	\[
	\left\|\frac{T(t)-I}{t}-\mathscr{A}\right\|
	\leq \|\mathscr{A}\|\cdot \max_{0\leq s\leq t}\|T(s)-I\|
	\]
	which imply that $T(t)$ is a uniformly continuous semigroup of bounded
	linear operators on $X$ and that $\mathscr{A}$ is its infinitesimal generator.
	
	\medskip
	
	Let $T(t)$ be a uniformly continuous semigroup of bounded linear operators
	on $X$. Fix $\rho>0$, small enough, such that
	\[
	\left\|I-\rho\int_{0}^{\rho}T(s)\,ds\right\|<1.
	\]
	This implies that $\rho^{-1}\int_{0}^{\rho}T(s)\,ds$ is invertible.
	Now,
	\[
	h^{-1}(T(h)-I)\int_{0}^{\rho}T(s)\,ds
	=
	h^{-1}\left(\int_{0}^{\rho}T(s+h)\,ds-\int_{0}^{\rho}T(s)\,ds\right)
	\]
	\[
	=
	h^{-1}\left(\int_{\rho}^{\rho+h}T(s)\,ds-\int_{0}^{h}T(s)\,ds\right)
	\]
	and therefore
	\[
	h^{-1}(T(h)-I)
	=
	\left(
	h^{-1}\int_{\rho}^{\rho+h}T(s)\,ds
	-
	h^{-1}\int_{0}^{h}T(s)\,ds
	\right)
	\left(\int_{0}^{\rho}T(s)\,ds\right)^{-1}.
	\tag{1.6}
	\]
	
	Letting $h\to 0$ in (1.6) shows that $h^{-1}(T(h)-I)$ converges in norm and
	therefore strongly to the bounded linear operator
	\[
	(T(\rho)-I)\left(\int_{0}^{\rho}T(s)\,ds\right)^{-1}
	\]
	which is the infinitesimal generator of $T(t)$.
\end{proof}
\begin{remark}
	\textcolor{blue}{The proof above was from the recommended text (Semigroups of Linear Operators and Applications to Partial Differential Equations) Page 2. [Theorem 1.2].}
\end{remark}

\begin{theo}\label{theorem 2.2.2}
Let $T(t)$ and $S(t)$ be uniformly continuous semigroups of
bounded linear operators. If
\[
\lim_{t\to 0}\frac{T(t)-I}{t}
=
\mathscr{A}
=
\lim_{t\to 0}\frac{S(t)-I}{t}.
\tag{1.7}
\]
then $T(t)=S(t)$ for $t\geq 0$.

\end{theo}

\begin{proof}
We will show that given $T>0$, $S(t)=T(t)$ for $0\leq t\leq T$.
Let $T>0$ be fixed, since $t\mapsto \|T(t)\|$ and $t\mapsto \|S(t)\|$
are continuous there is a constant $C$ such that
\[
\|T(t)\|\|S(s)\|\leq C \quad \text{for } 0\leq s,t\leq T.
\]
Given $\varepsilon>0$ it follows from (1.7) that there is a
$\delta>0$ such that
\[
h^{-1}\|T(h)-S(h)\|<\varepsilon/TC
\qquad \text{for } 0\leq h\leq \delta.
\tag{1.8}
\]

Let $0\leq t\leq T$ and choose $n\geq 1$ such that $t/n\leq \delta$.
From the semigroup property and (1.8) it then follows that
\begin{align*}
\|T(t)-S(t)\|
&=\left\|T\left(n\frac{t}{n}\right)-S\left(n\frac{t}{n}\right)\right\|\\
&\leq\sum_{k=0}^{n-1}
\left\|T\left((n-k)\frac{t}{n}\right)S\left(\frac{kt}{n}\right)-T\left((n-k-1)\frac{t}{n}\right)S\left(\frac{(k+1)t}{n}\right)\right\|\\
&\leq \sum_{k=0}^{n-1}\left\|T\left((n-k-1)\frac{t}{n}\right)\right\|\left\|T\left(\frac{t}{n}\right)-S\left(\frac{t}{n}\right) \right\|\left\|S\left(\frac{t}{n}\right)\right\|\leq Cn\frac{\varepsilon}{TC}\frac{t}{n}\leq \varepsilon.
\end{align*}

Since $\varepsilon>0$ was arbitrary $T(t)=S(t)$ for $0\leq t\leq T$
and the proof is complete. 
\end{proof}
\newpage
\begin{coro}\label{coro211}
Let $T(t)$ be a uniformly continuous semigroup of bounded linear
operators. Then

\begin{enumerate}
	\item[(a)]
	There exists a constant $\omega\geq 0$ such that $\|T(t)\|\leq e^{\omega t}$.
	\item[(b)]
	There exists a unique bounded linear operator $\A$ such that $T(t)=e^{t\mathscr{A}}$.
	\item[(c)]
	The operator $\mathscr{A}$ in part (b) is the infinitesimal generator of $T(t)$.
	\item[(d)]
	$t\mapsto T(t)$ is differentiable in norm and satisfies
	\[
	\frac{dT(t)}{dt}=\mathscr{A}T(t)=T(t)\mathscr{A}.
	\tag{1.9}
	\]
\end{enumerate}
\end{coro}

\begin{proof}
All the assertions of Corollary \ref{coro211} follow easily from (b).
To prove (b) note that the infinitesimal generator of $T(t)$ is a
bounded linear operator $\mathscr{A}$. $\mathscr{A}$ is also the infinitesimal generator
of $e^{t\mathscr{A}}$ defined by (1.5) and therefore, by Theorem \ref{theorem 2.2.2},
$T(t)=e^{t\mathscr{A}}$. 	
\end{proof}

\begin{remark}
	\textcolor{blue}{The proofs above are from the recommended text (Semigroups of Linear Operators and Applications to Partial Differential Equations) Page 3. [Theorem 1.3 and Corollary 1.4].}
\end{remark}

\section{\texorpdfstring{Strongly Continuous Semigroups ($C_0$-Semigroups)}{Strongly Continuous Semigroups}}

\begin{definition}\label{def:sc-sg}
	The SG-BLO $(T(t))_{t \ge 0}$ is strongly continous (SC or $C_0$) if $\forall x \in X$
	$$
		\norm{T(t)x - x}_{X} \xrightarrow{t \to 0^+} 0
	$$
\end{definition}

\begin{theo}\label{thm:sc-bdd}
	Let $(T(t))_{t \ge 0}$, $\text{C}_0\text{-SG}$ then $\exists \omega \ge 0, \exists M \ge 1, \forall t \ge 0 \, \norm{T(t)} \le Me^{\omega t}$
\end{theo}

\begin{proof}
	First we want to show that $\exists \eta > 0, \, \sup_{t \in [0, \eta]}{\norm{T(t)}} < +\infty$.

	By contradiction, assume that $\sup{\norm{T(t)}} = +\infty$. Then $\exists (t_n)_{n \ge 0} \searrow 0$ such that $\norm{T(t_n)} \ge n$ or $\norm{T(t_n)} \nearrow \infty$

	By Banach-Steinhaus (the contrapositive) $\exists x \in X$ such that $\sup{\norm{T(t_n)x}} = + \infty$, but this contradicts the strong convergance \ref{def:sc-sg}.

	Now take $M := \sup{\norm{T(t)}} \ge 1$ (This is because $T(0) = Id$).

	$\forall t \ge 0$ write $t = k\eta + \eta_t$ where $k = \left\lfloor{\frac{t}{\eta}}\right\rfloor$ and $\eta_t \in [0, \eta]$, then
	\begin{align*}
		\norm{T(t)} &= \norm{T(k\eta + \eta_t)} \\
			    &= \norm{\left[T(\eta)\right]^k T(\eta_t)} && \text{by the SG property} \\
			    &= \norm{T(\eta)}^k \norm{T(\eta_t)} \\
			    &\le M \cdot M^k && \text{M is upperbound}\\
			    &\le M \cdot M^{t/\eta} && \text{since } k \le \frac{t}{\eta} \\
			    &= M (e^{\ln{M}})^{t/\eta} \\
			    &= M e^{t {\frac{\ln{M}}{\eta}}} = M e^{\omega t}
	\end{align*}
	And since $\omega = \dfrac{\ln{M}}{\eta}$ and $M \ge 1$ then $\omega \ge 0$. 
\end{proof}

\begin{coro} \label{sc:cont}
Let $\{T(t)\}_{t \ge 0}$ be a $C_0$-semigroup then $\forall x \in X$, $t \mapsto T(t)x$ is continuous
\end{coro}

\begin{proof}
For $h > 0$ and $t > 0$:
\begin{itemize}
\item $T(t+h)x - T(t)x = T(t)[T(h) - Id]x \to 0$ as $h \to 0$ by definition \ref{def:sc-sg}.
\item $T(t-h)x - T(t)x = T(t-h)[Id - T(h)]x$. Since $\norm{T(t-h)}$ is bounded by theorem \ref{thm:sc-bdd}, so this tends to $0$ as $h \to 0$.
\end{itemize}
\end{proof}

\begin{theo*}
Let $\A$ be the IG of $C_0$-SG $T(t)$, then:

\begin{enumerate}[label=\bfseries \thesection.\arabic*, 
                 before=\setcounter{enumi}{\value{theo}}, 
                 after=\setcounter{theo}{\value{enumi}}]
	    \item \label{sc:space_cont} $\forall x \in X, \forall t \ge 0, \; \lim_{h \to 0} \frac{1}{h} \int_t^{t+h} T(s)x \, ds = T(t)x$ for all $x \in X$.
	    \item \label{sc:img} $\forall x \in X, \forall t \ge 0, \; \int_0^t T(s)x \, ds \in D(\A)$ and $\A \int_0^t T(s)x \, ds = T(t)x - x$.
	    \item \label{sc:diff_commute} $\forall x \in D(\A)$, $T(t)x \in D(\A)$ and $\frac{d}{dt} T(t)x = \A T(t)x = T(t)\A x$.
	    \item \label{sc:ftc} $\forall x \in D(\A), \forall t \ge 0, \forall s \ge 0, \; T(t)x - T(s)x = \int_s^t T(u)\A x \, du$ for $x \in D(\A)$.
\end{enumerate}
\end{theo*}

\begin{proof}[Proof of \ref{sc:space_cont}]
Consider the small interval $[t, t+h]$ relative to its value at $t$:
$$
    \frac{1}{h} \int_{t}^{t+h} T(s)x \, ds - T(t)x = \frac{1}{h} \int_{t}^{t+h} [T(s)x - T(t)x] \, ds \text{}
$$
By the continuity (corollary \ref{sc:cont}) of the map $s \mapsto T(s)x$, for any $\epsilon > 0$, there exists a $\delta > 0$ such that for all $s$ satisfying $|s - t| < \delta$, we have $\norm{T(s)x - T(t)x} < \epsilon$.

Taking $0 < h < \delta$, we can estimate the norm of the integral:
$$
	\norm{\frac{1}{h} \int_{t}^{t+h} [T(s)x - T(t)x] \, ds } \le \frac{1}{h} \int_{t}^{t+h} \norm{T(s)x - T(t)x} \, ds < \frac{1}{h} \cdot h\epsilon = \epsilon \text{}
$$
Then
$$
    \lim_{h \to 0} \frac{1}{h} \int_{t}^{t+h} T(s)x \, ds = T(t)x \text{}
$$
\end{proof}

\begin{proof}[Proof of \ref{sc:img}]
Let $h > 0$. Consider the difference quotient for the integral $y = \int_0^t T(s)x \, ds$:
\begin{align*}
\frac{T(h)-Id}{h} \int_0^t T(s)x \, ds &= \frac{1}{h} \left[ \int_0^t T(s+h)x \, ds - \int_0^t T(s)x \, ds \right] \\
&= \frac{1}{h} \left[ \int_h^{t+h} T(u)x \, du - \int_0^t T(u)x \, du \right] \\
&= \frac{1}{h} \int_t^{t+h} T(u)x \, du - \frac{1}{h} \int_0^h T(u)x \, du
\end{align*}
As $h \to 0^+$, the first term converges to $T(t)x$ and the second to $T(0)x = x$ by \ref{sc:space_cont}. Thus the limit exists, $y \in D(\A)$, and $\A y = T(t)x - x$.
\end{proof}

\begin{proof}[Proof of \ref{sc:diff_commute}]
If $x \in D(\A)$, then $T(t)\A x = T(t) \lim_{h \to 0} \frac{T(h)x-x}{h} = \lim_{h \to 0} \frac{T(h)T(t)x - T(t)x}{h}$. This limit exists and equals $\A(T(t)x)$, proving $T(t)x \in D(\A)$ and $T(t)\A x = \A T(t)x$. This also shows the right-derivative of $T(t)x$ is $\A T(t)x$. A similar argument for the left-derivative completes the differentiability.
\end{proof}

\begin{proof}[Proof of \ref{sc:ftc}]
By Property \ref{sc:diff_commute}, the function $f(u) = T(u)x$ is differentiable with $f'(u) = T(u)\A x$. Since $f'$ is continuous, we integrate $f'$ over $[s, t]$ to obtain $f(t) - f(s) = \int_s^t f'(u) \, du$, which is $T(t)x - T(s)x = \int_s^t T(u)\A x \, du$.
\end{proof}

\begin{theo} \label{sc:dense}
The IG $\A$ of $C_0$-SG is a closed linear operator and $\overline{D(\A)} = X$.
\end{theo}

\begin{proof}
For any $x \in X$, let $x_t = \frac{1}{t} \int_0^t T(s)x \, ds$. By \ref{sc:img}, $x_t \in D(\A)$. By \ref{sc:space_cont}, $x_t \to T(0)x = x$ as $t \to 0^+$, which shows $\overline{D(\A)} = X$.

Let $x_n \in D(\A)$ such that $x_n \to x$ and $\A x_n \to y$. From \ref{sc:img} take $s = 0$, we have $T(t)x_n - x_n = \int_0^t T(s)\A x_n \, ds$. Passing to the limit $n \to \infty$, we get $T(t)x - x = \int_0^t T(s)y \, ds$. Dividing by $t$ and letting $t \to 0^+$, the Right-Hand Side (RHS) converges to $y$. Thus $x \in D(\A)$ and $\A x = y$.
\end{proof}

\begin{theo}
	Let $\{T(t)\}_{t \ge 0}$ and $\{S(t)\}_{t \ge 0}$ be two $C_0$-SG with infinitesimal generators $\A$ and $B$, respectively. If $\A = \mathscr{B}$, then $T(t) = S(t)$ for all $t \ge 0$.
\end{theo}

\begin{proof}
	Assume $\A = \mathscr{B}$, then $D(\A) = D(\mathscr{B})$. Let $x \in D(\A)$ be fixed, and for a fixed $t > 0$, define
$$
    \varphi: [0, t] \to X, \quad \varphi(s) = T(t-s)S(s)x
$$
Since $x \in D(\A)$, the map $\varphi$ is of class $C^1$ on $[0, t]$. We differentiate $\varphi$ with respect to $s$ using the product rule and \ref{sc:diff_commute}, we get:
\begin{align*}
    \frac{d}{ds} \varphi(s) &= \frac{d}{ds} [T(t-s)] S(s)x + T(t-s) \frac{d}{ds} [S(s)x] \\
    &= -\A T(t-s)S(s)x + T(t-s) B S(s)x
\end{align*}
Because $T(t-s)$ commutes with its generator $\A$, and given $\A = B$, we have:
$$
    \frac{d}{ds} \varphi(s) = -T(t-s)\A S(s)x + T(t-s)\A S(s)x = 0
$$
Since the derivative is zero for all $s \in [0, t]$, the function $\varphi$ must be constant. Evaluating $\varphi$ at the endpoints $s=0$ and $s=t$ yields
$$
    \varphi(0) = T(t)S(0)x = T(t)x \quad \text{and} \quad \varphi(t) = T(0)S(t)x = S(t)x
$$
Thus, $T(t)x = S(t)x$ for all $x \in D(\A)$. Since $D(\A)$ is dense in $X$ and $T(t), S(t)$ are bounded linear operators, this identity extends to all $x \in X$ by continuity. Therefore, $T(t) = S(t)$ for all $t \ge 0$.
\end{proof}

\begin{theo}
Let $\A$ be the IG of a $C_0$-SG $\{T(t)\}_{t \ge 0}$ on a Banach space $X$. Then the subspace
$$
	X = \overline{\bigcap_{n \ge 1} D(\A^n)}
$$
\end{theo}

\begin{proof}
Let $\mathscr{D} = \left\{\varphi:\mathbb{R} \to \mathbb{C} | \, \varphi \, \text{has compact support in } \mathbb{R}^{*}_{+} \, \text{and is smooth }C^{\infty} \right\}$. Let $x \in X$ and consider a test function $\varphi \in \mathscr{D}$. Define
$$
    x_\varphi = \int_0^\infty \varphi(s) T(s)x \, ds
$$

First, we show that $x_\varphi \in D(\A)$. Consider
\begin{align*}
    \frac{T(h)-Id}{h} x_\varphi &= \frac{1}{h} \int_0^\infty \varphi(s) [T(s+h)x - T(s)x] \, ds \\
    &= \frac{1}{h} \left[ \int_h^\infty \varphi(u-h) T(u)x \, du - \int_0^\infty \varphi(u) T(u)x \, du \right] \\
    &= \int_0^\infty \frac{\varphi(u-h) - \varphi(u)}{h} T(u)x \, du
\end{align*}
As $h \to 0$, the quotient $\frac{\varphi(u-h) - \varphi(u)}{h}$ converges uniformly to $-\varphi'(u)$ because $\varphi$ is $C^{\infty}$ and has compact support. Thus:
$$
    \A x_\varphi = - \int_0^\infty \varphi'(s) T(s)x \, ds
$$
Since $\varphi' \in C_c^\infty(0, \infty)$, we can repeat this process inductively. For any $n \ge 1$, we find:
$$
    \A^n x_\varphi = (-1)^n \int_0^\infty \varphi^{(n)}(s) T(s)x \, ds
$$
This proves that $x_\varphi \in D(\A^n)$ for all $n$.

To prove density, suppose $\overline{\bigcap_{n \ge 1} D(\A^n)} \neq X$. By the Hahn-Banach Theorem, there exists a non-zero functional $x^* \in X^*$ such that $\langle x^*, y \rangle = 0$ for all $y \in \bigcap_{n \ge 1} D(\A^n)$. Specifically, for any $x \in X$ and $\varphi \in C_c^\infty(0, +\infty)$:
$$
	\langle x^*, x_\varphi \rangle = \int_0^{+\infty} \varphi(s) \langle x^*, T(s)x \rangle \, ds = 0
$$
This identity holds for all $C^{\infty}$ functions $\varphi$ with compact support. Then $\langle x^*, T(s)x \rangle$ must be zero for all $s > 0$. 

By the strong continuity of the semigroup at $s=0$:
$$
    \langle x^*, x \rangle = \lim_{s \to 0^+} \langle x^*, T(s)x \rangle = 0
$$
Since this holds for all $x \in X$, it implies $x^* = 0$, which contradicts our assumption that $x^*$ was non-zero. Thus, $\bigcap_{n \ge 1} D(\A^n)$ must be dense in $X$.
\end{proof}

\begin{exercise}{}{}
Let $X = \left\{ f: \mathbb{R} \to \mathbb{C} \, | \, f \, \text{is continuous and uniformly bounded}\right\}$ equipped with the supremum norm $\norm{f}_\infty = \sup_{s \in \mathbb{R}} |f(s)|$. Define the family of operators $(T(t))_{t \ge 0}$ by:
$$
    (T(t)f)(s) = f(s+t), \quad s \in \mathbb{R}, t \ge 0
$$
Prove that this family is $C_0$-SG, its IG is $\A f = f'$, and $\norm{T(t)} = 1$.
\end{exercise}

\section{Hille-Yosida Theorem}

\begin{definition}
A $C_0$-SG $\{T(t)\}_{t \ge 0}$ is called uniformly bounded semigroup if $\exists M \ge 1$ such that $\norm{T(t)} \le M$ for all $t \ge 0$.
\end{definition}

\begin{definition}
A $C_0$-SG $\{T(t)\}_{t \ge 0}$ is called a contraction semigroup if $\norm{T(t)} \le 1$ for all $t \ge 0$.
\end{definition}

\begin{theo}[Hille-Yosida Theorem (Contraction Case)]
A linear unbounded operator $\A: D(\A) \subset X \to X$ is the IG of a $C_0$-SG of contractions if and only if:
\begin{enumerate}[label=(\roman*)]
	\item \label{hy:close} $\A$ is closed and $\overline{D(\A)} = X$.
	\item \label{hy:bound} $\mathbb{R}^*_+ \subset \rho(\A)$ and $\norm{R(\lambda, \A)} \le \frac{1}{\lambda}$ for all $\lambda > 0$.
\end{enumerate}
\end{theo}

\begin{proof}[($\Rightarrow$)]
\ref{hy:close} follows directly from \ref{sc:dense}.

If $\A$ generates a contraction semigroup $\{T(t)\}_{t \ge 0}$, we define the resolvent for $\lambda > 0$ as follows
$$
R(\lambda) = \int_0^{+\infty} e^{-\lambda t} T(t) \, dt
$$
Taking the norm, we obtain:
$$
\norm{R(\lambda)x} \le \int_0^{+\infty} e^{-\lambda t} \norm{T(t)x} \, dt \le \int_0^{+\infty} e^{-\lambda t} \norm{x} \, dt = \frac{1}{\lambda}\norm{x}
$$
\end{proof}

\begin{remark}
Note for all real numbers $\lambda, a$ with $\lambda > a$, we have the following:
$$
\frac{1}{\lambda - a} = \int\limits^{+\infty}_{0} {e^{-(\lambda - a)t} \, dt}
$$
Extending this to the vector space we get the way of writing the resolvent operator from above.
\end{remark}

\subsection{The Yosida Approximation}

To prove if ($\Leftarrow$), we introduce a family of bounded operators that approximate the unbounded generator $A$.

\begin{definition}
\label{hy:yoshida_approx} For $\lambda > 0$, the Yosida Approximation of $\A$ is defined as:
$$
        \A_\lambda := \lambda \A R(\lambda, \A) = \lambda^2 R(\lambda, \A) - \lambda I
$$
\end{definition}
Note that $\A_\lambda$ is a bounded linear operator for each $\lambda \in \rho(\A)$.

\begin{claim}
For $\lambda \in \rho(\A)$ and $x \in D(\A)$, the following identity holds:
$$
    \lambda R(\lambda, \A)x - x = \A R(\lambda, \A)x = R(\lambda, \A)\A x
$$
\end{claim}

\begin{proof}
By the definition of the resolvent as the inverse of the operator $(\lambda I - \A)$, we have:
$$
(\lambda I - \A) R(\lambda, \A) = Id_{X}
$$
Applying this to any $x \in X$:
$$
    (\lambda I - \A) R(\lambda, \A)x = x
$$
Distributing the operators on the left-hand side gives:
$$
    \lambda R(\lambda, \A)x - \A R(\lambda, \A)x = x
$$
Rearranging the terms to isolate the $\A R(\lambda, \A)x$ term yields:
\begin{equation*}
\lambda R(\lambda, \A)x - x = \A R(\lambda, \A)x \tag{*}
\end{equation*}
This identity holds for all $x \in X$ because $R(\lambda, \A)$ maps $X$ into $D(\A)$.

For the other side, let $x \in D(\A)$. We use the fact that the resolvent also satisfies:
$$
	R(\lambda, \A) (\lambda I - \A) = Id_{D(\A)}
$$
Applying this to $x \in D(\A)$:
$$
    R(\lambda, \A) (\lambda I - \A)x = x
$$
Distributing $R(\lambda, \A)$ gives:
$$
    \lambda R(\lambda, \A)x - R(\lambda, \A)\A x = x
$$
Rearranging the terms:
\begin{equation*}
    \lambda R(\lambda, \A)x - x = R(\lambda, \A)\A x \tag{**}
\end{equation*}

From (*) and (**) we get:
$$
    \A R(\lambda, \A)x = R(\lambda, \A)\A x
$$
\end{proof}

\begin{theo}
\label{hy:yosida_conv} For all $x \in X$, $\displaystyle \lim_{\lambda \to +\infty} \A_\lambda x = \A x$.
\end{theo}

\begin{proof}
Using the identity $\lambda R(\lambda, \A)x - x = \A R(\lambda, \A)x = R(\lambda, \A)\A x$ for $x \in D(\A)$, we observe:
$$
\norm{\lambda R(\lambda, \A)x - x } = \norm{ R(\lambda, \A)\A x } \le \frac{\norm{\A x}}{\lambda} \xrightarrow{\lambda \to +\infty} 0
$$
Since $\A_\lambda x = \lambda R(\lambda, \A)\A x$, and we just showed $\lambda R(\lambda, \A)x \to x, \; \forall x \in D(\A)$, it follows that $\A_\lambda x \to \A x$. By density and the uniform bound $\norm{\lambda R(\lambda, \A)} \le 1$, this convergence holds for all $x \in X$.
\end{proof}

\begin{lemma}
\label{hy:yosida_sg} For each $\lambda > 0$, $\A_\lambda$ generates a uniformly continuous semigroup of contractions $\{e^{t\A_\lambda}\}_{t \ge 0}$.
\end{lemma}
\begin{proof}
Since $\A_\lambda = \lambda^2 R(\lambda, \A) - \lambda I$, we have:
$$
	\norm{e^{t\A_\lambda}} = \norm{e^{-t\lambda} e^{t\lambda^2 R(\lambda, \A)}} \le e^{-t\lambda} e^{t\lambda^2 \norm{R(\lambda, \A)}} \le e^{-t\lambda} e^{t\lambda^2 \frac{1}{\lambda}} = e^{-t\lambda} e^{t\lambda} = 1
$$
This confirms the contraction property for the approximating semigroups.
\end{proof}

\begin{lemma} \label{hy:uc}
For any $x \in X$ and $\lambda, \mu > 0$, the following estimate holds:
$$
\norm{e^{t\A_\lambda}x - e^{t\A_\mu}x} \le t \norm{\A_\lambda x - \A_\mu x}
$$
\end{lemma}

\begin{proof}
Fix $x \in X$, consider the function $\phi: [0, 1] \to X$ defined by:
$$
\phi(s) = e^{s t \A_\lambda} e^{(1-s) t \A_\mu} x
$$
Since $\A_\lambda$ and $\A_\mu$ commute (as resolvents commute), the semigroups $e^{t\A_\lambda}$ and $e^{t\A_\mu}$ also commute. The function $\phi$ is differentiable with respect to $s$:
\begin{align*}
\frac{d}{ds} \phi(s) &= t \A_\lambda e^{s t \A_\lambda} e^{(1-s) t \A_\mu} x - e^{s t \A_\lambda} t \A_\mu e^{(1-s) t \A_\mu} x \\
&= t e^{s t \A_\lambda} e^{(1-s) t \A_\mu} (\A_\lambda - \A_\mu) x
\end{align*}
Integrating from $0$ to $1$:
$$
\phi(1) - \phi(0) = e^{t \A_\lambda}x - e^{t \A_\mu}x = \int_0^1 t e^{s t \A_\lambda} e^{(1-s) t \A_\mu} (\A_\lambda - \A_\mu) x \, ds
$$
Taking the norm and using the contraction property $\|e^{t \A_\lambda}\| \le e^{t \|\A_\lambda\|} \le 1$ (since $\A_\lambda$ is dissipative):
\begin{align*}
\norm{e^{t \A_\lambda}x - e^{t \A_\mu}x} &\le \int_0^1 t \norm{e^{s t \A_\lambda}} \norm{e^{(1-s) t \A_\mu}} \norm{\A_\lambda x - \A_\mu x} \, ds \\
&\le t \norm{\A_\lambda x - \A_\mu x}
\end{align*}
\end{proof}

\bigskip

Now let's start proving the only if direction of Hille-Yosida ($\impliedby$).

\begin{proof}
We need the sequence $(e^{t\A_\lambda}x)_{\lambda > 0}$ to be converging. But doing so is hard, what we can do is show that it is a Cauchy sequence in $X$, since $X$ is complete.

Taking $x \in D(\A)$ we have the following:
$$
	\norm{e^{t\A_\lambda}x - e^{t\A_\mu}x} \le t\norm{\A_\lambda x - \A_\mu x} \le t(\norm{\A_\lambda x - \A x} + \norm{\A x - \A_\mu x}) \xrightarrow{\substack{\lambda \to +\infty \\ \mu \to +\infty}} 0
$$

This is possible because $\lim_{\lambda \to +\infty} {\A_\lambda x} = \A x$.

Define:
$$
	T(t)x = \lim_{\lambda \to +\infty} {e^{t\A_\lambda}x}
$$

This limit is well defined by the argument above for all $x \in D(\A)$. Moreover, since $D(A)$ is dense in $X$ and $\|e^{tA_\lambda}\| \le 1$, we can extend $T(t)$ to a bounded linear operator on all of $X$ by density.

Now we show that the family $(T(t))_{t \ge 0}$ defined above is a $C_0$-semigroup of contractions.

We verify the semigroup properties:
\begin{enumerate}
    \item Identity: $T(0)x = \lim_{\lambda \to +\infty} e^{0 \cdot \A_\lambda}x = x$.
    \item Semigroup Property: For $x \in X$,
    $$
    T(t+s)x = \lim_{\lambda \to +\infty} e^{(t+s)\A_\lambda}x = \lim_{\lambda \to +\infty} e^{t\A_\lambda} e^{s\A_\lambda}x = T(t)T(s)x.
    $$
\item Strong Continuity: For $x \in D(\A)$, convergence is uniform on compact intervals of $t$, because of lemma \ref{hy:uc}. Thus $t \mapsto T(t)x$ is continuous. By the density of $D(A)$ and uniform boundedness $\|T(t)\| \le 1$, continuity extends to all $x \in X$.
\end{enumerate}

We must show that the generator of the constructed semigroup $T(t)$ is indeed $\A$. Let $\mathscr{B}$ be the generator of $T(t)$, we show that $\A = \mathscr{B}$.

For any $x \in D(\A)$, we have the identity:
$$
e^{t\A_\lambda}x - x = \int_0^t e^{s\A_\lambda} \A_\lambda x \, ds
$$
As $\lambda \to +\infty$, $e^{s\A_\lambda} \to T(s)$ strongly and uniformly on compact sets, and $\A_\lambda x \to \A x$. Passing to the limit:
$$
T(t)x - x = \int_0^t T(s) \A x \, ds
$$
Dividing by $t$ and taking $t \to 0^+$:
$$
\lim_{t \to 0^+} \frac{T(t)x - x}{t} = \lim_{t \to 0^+} \frac{1}{t} \int_0^t T(s) \A x \, ds = T(0)\A x = \A x
$$
Thus, $x \in D(\mathscr{B})$ and $\mathscr{B}x = \A x$, implying $\A \subset \mathscr{B}$.
Since $\mathscr{B}$ is the generator of a $C_0$-semigroup of contractions, $1 \in \rho(\mathscr{B})$. By hypothesis, $1 \in \rho(\A)$. Since $\A \subset \mathscr{B}$ and both $(I-\A)$ and $(I-\mathscr{B})$ are surjective (mapping onto $X$), it follows that $\A = \mathscr{B}$.
\end{proof}

\begin{coro} \label{ya:limit}
Let $\A$ be the IG of a $C_0$-SG of contractions $(T(t))_{t \ge 0}$. Then for every $x \in X$, the semigroup is given by the limit:
$$
T(t)x = \lim_{\lambda \to +\infty} e^{t \A_\lambda} x
$$
\end{coro}

\begin{proof}
In the previous proof, we constructed a SG, let us call it $S(t)$, defined by $S(t)x = \lim_{\lambda \to +\infty} e^{t \A_\lambda}x$. We proved that the generator of $S(t)$ is exactly the operator $\A$.

Since $\A$ is the generator of the original semigroup $T(t)$ by hypothesis, and we know that a $C_0$-SG is uniquely determined by its generator (Uniqueness Theorem), it follows that:
$$
T(t) = S(t), \quad \forall t \ge 0
$$
\end{proof}

\begin{coro}
Let $\A$ be IG of a $C_0$-SG of contractions $(T(t))_{t \ge 0}$. Then the resolvent set $\rho(\A)$ contains the open right half-plane:
$$
\rho(\A) \supset \{ \lambda \in \mathbb{C} \mid \Re(\lambda) > 0 \}
$$
Furthermore, for all $\lambda$ with $\Re(\lambda) > 0$, the following estimate holds:
$$
\norm{R(\lambda, \A)} \le \frac{1}{\Re(\lambda)}
$$
\end{coro}

\begin{proof}
Let $\lambda \in \mathbb{C}$ with $\Re(\lambda) > 0$. We define the operator $R(\lambda)$ on $X$ by the Laplace transform of the semigroup:
$$
R(\lambda)x = \int_0^{+\infty} e^{-\lambda t} T(t)x \, dt, \quad \forall x \in X
$$
Since $T(t)$ is a contraction semigroup ($\|T(t)\| \le 1$) and $\Re(\lambda) > 0$, the integrand is exponentially bounded:
$$
\norm{e^{-\lambda t} T(t)x} = e^{-\Re(\lambda)t} \norm{T(t)x} \le e^{-\Re(\lambda)t} \norm{x}
$$
Thus, the integral converges absolutely, defining a bounded linear operator. We calculate its norm:
$$
\norm{R(\lambda)x} \le \int_0^{+\infty} e^{-\Re(\lambda)t} \norm{x} \, dt = \norm{x} \left[ \frac{-e^{-\Re(\lambda)t}}{\Re(\lambda)} \right]_0^{+\infty} = \frac{1}{\Re(\lambda)} \norm{x}
$$
This proves the bound $\|R(\lambda)\| \le \dfrac{1}{\Re(\lambda)}$.

It remains to show that this integral operator $R(\lambda)$ is indeed the resolvent $R(\lambda, A) = (\lambda I - A)^{-1}$.
For any $x \in X$ and $h > 0$:
\begin{align*}
	\frac{T(h) - I}{h} R(\lambda)x &= \frac{1}{h} \int_0^{+\infty} e^{-\lambda t} T(t+h)x \, dt - \frac{1}{h} \int_0^{+\infty} e^{-\lambda t} T(t)x \, dt \\
				       &= \frac{e^{\lambda h}}{h} \int_h^{+\infty} e^{-\lambda s} T(s)x \, ds - \frac{1}{h} \int_0^{+\infty} e^{-\lambda t} T(t)x \, dt \\
				       &= \frac{e^{\lambda h} - 1}{h} \int_0^{+\infty} e^{-\lambda t} T(t)x \, dt - \frac{e^{\lambda h}}{h} \int_0^h e^{-\lambda t} T(t)x \, dt
\end{align*}
Taking the limit as $h \to 0^+$, the first term converges to $\lambda R(\lambda)x$ and the second term converges to $-x$.
Thus, for any $x \in X$, $R(\lambda)x \in D(\A)$ and $\A R(\lambda)x = \lambda R(\lambda)x - x$, which implies $(\lambda I - \A)R(\lambda)x = x$. Similarly, one can show $R(\lambda)(\lambda I - \A)x = x$ for $x \in D(\A)$.

Therefore, $R(\lambda) = (\lambda I - \A)^{-1}$.
\end{proof}

\begin{exercise}{}{}
Let $X = BVC(\mathbb{R}_+) = \{ f: \mathbb{R}_+ \to \mathbb{C} \mid f \text{ is bounded and uniformly continuous} \}$.
Equipped with the supremum norm $\norm{f}_\infty = \sup_{t \ge 0} |f(t)|$, $(X, \norm{\cdot}_\infty)$ is a Banach space.

For $t \ge 0$, define the operator $T(t)$ by the left shift:
$$
(T(t)f)(s) = f(s+t), \quad \forall s \ge 0
$$
Check the following:
\begin{enumerate}
    \item $(T(t))_{t \ge 0}$ is a $C_0$-semigroup of contractions (i.e., $\|T(t)\| \le 1$).
    \item Show that $\norm{T(t)} = 1$.
    \item The infinitesimal generator is the differentiation operator $\A f = f'$, with an appropriate domain $D(\A)$.
    \item Verify that $\rho(\A) \supset \{ \lambda \in \mathbb{C} \mid \Re(\lambda) > 0 \}$.
\end{enumerate}
\end{exercise}

\subsection{The General Hille-Yosida Theorem}

We now consider the general case where the semigroup is not necessarily a contraction.

\begin{theo}
A linear operator $\A$ generates a $C_0$-semigroup $T(t)$ satisfying $\norm{T(t)} \le M e^{\omega t}$ if and only if:
\begin{enumerate}[label=(\roman*)]
    \item $A$ is closed and densely defined.
    \item $(\omega, +\infty) \subset \rho(\A)$ and for all $\lambda > \omega$ and $n \ge 1$:
    $$
    \norm{R(\lambda, A)^n} \le \frac{M}{(\lambda - \omega)^n}.
    $$
\end{enumerate}
\end{theo}

\subsection{\texorpdfstring{Reduction to the Case $\omega = 0$}{Reduction to the case of uniformly bounded}}
If $\norm{T(t)} \le M e^{\omega t}$, consider the rescaled semigroup $S(t) = e^{-\omega t}T(t)$. The generator of $S(t)$ is $\A - \omega I$ where $\A$ is the IG of $\left(T(t)\right)_{t \ge 0}$, and $\norm{S(t)} \le M$. Conversely, if we prove the theorem for $\omega = 0$, the general case follows by applying the result to $\A - \omega I$.

Now we have the following corollary:

\begin{coro}[Hille-Yosida for $(1, \omega)$]
A linear operator $\A$ is the IG of a $C_0$-SG satisfying $\norm{T(t)} \le e^{\omega t}$ if and only if:
\begin{enumerate}[label=(\roman*)]
    \item $\A$ is closed and $\overline{D(\A)} = X$.
    \item $\rho(\A) \supset (\omega, +\infty)$ and for all $\lambda > \omega$, the following estimate holds:
    $$
    \norm{R(\lambda, \A)} \le \frac{1}{\lambda - \omega}.
    $$
\end{enumerate}
\end{coro}

\begin{proof}
\noindent $(\Rightarrow)$:
Suppose $\A$ generates a semigroup $T(t)$ such that $\norm{T(t)} \le e^{\omega t}$. Consider the rescaled family of operators $S(t) = e^{-\omega t} T(t)$. It is easy to verify that $S(t)$ is a $C_0$-SG. Furthermore, it is a contraction:
$$
\norm{S(t)} = e^{-\omega t} \norm{T(t)} \le e^{-\omega t} e^{\omega t} = 1
$$
Let $\mathscr{B}$ be the generator of $S(t)$. By the definition of the generator:
$$
\mathscr{B}x = \lim_{t \to 0^+} \frac{e^{-\omega t} T(t)x - x}{t} = \lim_{t \to 0^+} \left( e^{-\omega t} \frac{T(t)x - x}{t} + \frac{e^{-\omega t} - 1}{t} x \right) = \A x - \omega x
$$
Thus, $\mathscr{B} = \A - \omega I$.
Since $\mathscr{B}$ generates a contraction semigroup, by the Hille-Yosida Theorem for contractions (Case $M=1, \omega=0$), we know that for any $\mu > 0$, $\mu \in \rho(\mathscr{B})$ and $\norm{R(\mu, \mathscr{B})} \le \dfrac{1}{\mu}$.

Let $\lambda = \mu + \omega$. Then $\lambda > \omega$. Since $R(\mu, \mathscr{B}) = (\mu I - \mathscr{B})^{-1} = (\mu I - (\A - \omega I))^{-1} = ((\mu + \omega)I - \A)^{-1}$, we have:
$$
	R(\lambda, \A) = R(\lambda - \omega, \mathscr{B})
$$
Substituting the norm bound:
$$
	\norm{R(\lambda, \A)} = \norm{R(\lambda - \omega, \mathscr{B})} \le \frac{1}{\lambda - \omega}
$$

\noindent $(\Leftarrow)$:
Conversely, suppose $\A$ satisfies conditions (1)-(3). Define $\mathscr{B} = \A - \omega I$.
Clearly, $\mathscr{B}$ is closed and densely defined.
For any $\mu > 0$, let $\lambda = \mu + \omega > \omega$. Then $\lambda \in \rho(\A)$, which implies $\mu \in \rho(\mathscr{B})$.
The resolvent satisfies:
$$
	\norm{R(\mu, \mathscr{B})} = \norm{R(\mu + \omega, \A)} \le \frac{1}{(\mu + \omega) - \omega} = \frac{1}{\mu}
$$
Thus, $\mathscr{B}$ satisfies the Hille-Yosida conditions for the contraction case ($M=1, \omega=0$). Therefore, $\mathscr{B}$ generates a contraction semigroup $S(t)$ with $\norm{S(t)} \le 1$.
Defining $T(t) = e^{\omega t} S(t)$, we see that $T(t)$ is a $C_0$-semigroup generated by $\A = \mathscr{B} + \omega I$, and it satisfies:
$$
\norm{T(t)} = e^{\omega t} \norm{S(t)} \le e^{\omega t}
$$
\end{proof}

This corollary provides us with a method to rescale the bound. Thus, we focus on the case $(M, 0)$, i.e., $\norm{T(t)} \le M$.
The condition on the resolvent becomes $\norm{R(\lambda,\A)^n} \le \dfrac{M}{\lambda^n}$.

\begin{lemma} \label{sc:mnbound}
	Let $(T(t))_{t \ge 0}$ be a $C_0$-semigroup satisfying $\norm{T(t)} \le M$ for all $t \ge 0$. Let $\A$ be its infinitesimal generator. Then for all $\lambda > 0$ and all integers $n \ge 0$:
$$
\norm{R(\lambda, \A)^n} \le \frac{M}{\lambda^n}
$$
Equivalently, $\norm{\lambda^n R(\lambda, \A)^n} \le M$.
\end{lemma}

\begin{proof}
For $\lambda > 0$, the resolvent is given by the Laplace transform of the semigroup:
$$
	R(\lambda, \A)x = \int_0^{+\infty} e^{-\lambda t} T(t)x \, dt, \quad \forall x \in X
$$
Since the integral converges absolutely (due to the exponential decay $e^{-\lambda t}$ and bounded $T(t)$), we can differentiate this expression with respect to $\lambda$ inside the integral sign. Differentiating $n-1$ times:
$$
	\frac{d^{n-1}}{d\lambda^{n-1}} R(\lambda, \A)x = \int_0^{+\infty} (-t)^{n-1} e^{-\lambda t} T(t)x \, dt
$$
On the other hand, from the general theory of resolvents, we have the identity:
$$
\frac{d^{k}}{d\lambda^{k}} R(\lambda, \A) = (-1)^k k! R(\lambda, \A)^{k+1}
$$
Setting $k = n-1$, we get:
$$
\frac{d^{n-1}}{d\lambda^{n-1}} R(\lambda, \A) = (-1)^{n-1} (n-1)! R(\lambda, \A)^n
$$
Equating the two expressions for the derivative:
$$
	(-1)^{n-1} (n-1)! R(\lambda, \A)^n x = \int_0^{+\infty} (-1)^{n-1} t^{n-1} e^{-\lambda t} T(t)x \, dt
$$
Simplifying and solving for $R(\lambda, \A)^n x$:
$$
	R(\lambda, \A)^n x = \frac{1}{(n-1)!} \int_0^{+\infty} t^{n-1} e^{-\lambda t} T(t)x \, dt
$$
Now, we take the norm and use the bound $\|T(t)\| \le M$:
\begin{align*}
	\norm{R(\lambda, \A)^n x} &\le \frac{1}{(n-1)!} \int_0^{+\infty} t^{n-1} e^{-\lambda t} \norm{T(t)x} \, dt \\
				  &\le \frac{M \norm{x}}{(n-1)!} \int_0^{+\infty} t^{n-1} e^{-\lambda t} \, dt
\end{align*}
The integral on the right is the Gamma function definition. Substituting $u = \lambda t$ ($dt = du/\lambda$):
$$
	\int_0^{+\infty} t^{n-1} e^{-\lambda t} \, dt = \frac{1}{\lambda^n} \int_0^{+\infty} u^{n-1} e^{-u} \, du = \frac{\Gamma(n)}{\lambda^n} = \frac{(n-1)!}{\lambda^n}
$$
Substituting this back into the inequality:
$$
\norm{R(\lambda, \A)^n x} \le \frac{M \norm{x}}{(n-1)!} \cdot \frac{(n-1)!}{\lambda^n} = \frac{M}{\lambda^n} \norm{x}
$$
Thus, $\norm{R(\lambda, \A)^n} \le \dfrac{M}{\lambda^n}$.
\end{proof}

\subsection{Renorming Lemma}
The idea is to construct an equivalent norm on $X$ under which $\A$ becomes dissipative (generating a contraction semigroup), allowing us to apply the contraction case result.

\begin{lemma} \label{re:norm}
	Let $A: D(A) \subset X \to X$ be a linear operator with $\rho(A) \supset \mathbb{R}_{+}^{*}$ such that for all $\lambda > 0$ and $n \ge 0$, $\|\lambda^n R(\lambda, A)^n\| \le M$.
Then, there exists a norm $\|\cdot\|_\mu$ on $X$ such that:
\begin{enumerate}
    \item The norms are equivalent: $\|x\| \le \|x\|_\mu \le M \|x\|$ for all $x \in X$.
    \item For all $\lambda > 0$, the operator $\lambda R(\lambda, A)$ is a contraction in the new norm: $\|\lambda R(\lambda, A) x\|_\mu \le \|x\|_\mu$.
\end{enumerate}
\end{lemma}

\begin{proof}
	Fix $\mu > 0$. We define the new norm $\norm{\cdot}_\mu$ by:
$$
	\norm{x}_\mu = \sup_{n \ge 0} \norm{\mu^n R(\mu, A)^n x}
$$

To show that the norms are equivalent, let $n=0$, the term is $\norm{I x} = \norm{x}$ so $\norm{x} \le \norm{x}_\mu$.
Using the hypothesis $\norm{\mu^n R(\mu, A)^n} \le M$, we have:
$$
	\norm{x}_\mu = \sup_{n \ge 0} \norm{\mu^n R(\mu, A)^n x} \le \sup_{n \ge 0} M \norm{x} = M \norm{x}
$$
Thus, $\norm{x} \le \norm{x}_\mu \le M \norm{x}$.

Now to show that $\mu R(\mu, A)$ is a contraction, we check the contraction property for the specific value $\mu$:
\begin{align*}
	\norm{\mu R(\mu, A) x}_\mu &= \sup_{n \ge 0} \norm{\mu^n R(\mu, A)^n (\mu R(\mu, A) x)} \\
				   &= \sup_{n \ge 0} \norm{\mu^{n+1} R(\mu, A)^{n+1} x} \\
				   &= \sup_{k \ge 1} \norm{\mu^k R(\mu, A)^k x} \\
				   &\le \sup_{k \ge 0} \norm{\mu^k R(\mu, A)^k x} = \norm{x}_\mu.
\end{align*}
So, $\norm{\mu R(\mu, A)}_\mu \le 1$.

Finally, we show the contraction for $0 < \lambda \le \mu$:

Let $x \in X$ and define $y = R(\lambda, \A)x$. We want to show $\lambda \norm{y}_\mu \le \norm{x}_\mu$.
Recall the Resolvent Identity:
$$
R(\lambda, \A) - R(\mu, \A) = (\mu - \lambda) R(\mu, \A) R(\lambda, \A)
$$
Applying this to $x$:
$$
y = R(\mu, \A)x + (\mu - \lambda) R(\mu, \A)y = R(\mu, \A) [x + (\mu - \lambda)y].
$$
Taking the $\norm{\cdot}_\mu$ norm and using the fact that $\norm{\mu R(\mu, A) z}_\mu \le \norm{z}_\mu \implies \norm{R(\mu, A) z}_\mu \le \frac{1}{\mu} \norm{z}_\mu$:
\begin{align*}
	\norm{y}_\mu &= \norm{R(\mu, A) [x + (\mu - \lambda)y]}_\mu \\
		     &\le \frac{1}{\mu} \norm{x + (\mu - \lambda)y}_\mu \\
		     &\le \frac{1}{\mu} \left( \norm{x}_\mu + (\mu - \lambda)\norm{y}_\mu \right)
\end{align*}
Multiplying by $\mu$:
$$
	\mu \norm{y}_\mu \le \norm{x}_\mu + (\mu - \lambda) \norm{y}_\mu
$$
Subtracting $(\mu - \lambda)\norm{y}_\mu$ from both sides (since $\mu - \lambda \ge 0$):
$$
\lambda \norm{y}_\mu \le \norm{x}_\mu.
$$
Thus, $\norm{\lambda R(\lambda, A)x}_\mu \le \norm{x}_\mu$ for $0 < \lambda \le \mu$.
Since this holds for any sufficiently large $\mu$, and the definition of the norm can be adjusted, this property extends to all $\lambda > 0$.
\end{proof}

\begin{remark}
In equivalent norm, the following are presarved:
\begin{enumerate}
	\item The closeness of an operator.
	\item The density of the image of an operator.
	\item The strong continuity of the $C_0$-SG.
\end{enumerate}
\end{remark}

\begin{theo}[Hille-Yosida for $(M, 0)$] \label{thm:HY_M0}
	Let $\A: D(\A) \subset X \to X$ be a linear operator on a Banach space $X$. Then $\A$ is the IG of a $C_0$-SG $T(t)$ satisfying $\norm{T(t)} \le M$ for all $t \ge 0$ if and only if:
\begin{enumerate}[label=(\roman*)]
	\item \label{hym0:closed} $\A$ is closed and ($\overline{D(\A)} = X$).
	\item \label{hym0:bound} $\rho(\A) \supset (0, \infty)$ and for every $\lambda > 0$:
    $$
	    \norm{(\lambda^n R(\lambda, \A))^n} \le M, \quad \forall n \in \mathbb{N}
    $$
\end{enumerate}
\end{theo}

\begin{proof}
$\Rightarrow$ \\
Let $\A$ be the generator of a $C_0$-semigroup $T(t)$ with $\norm{T(t)} \le M$. By theorem \ref{sc:dense} part \ref{hym0:closed} is done.

For part \ref{hym0:bound} is direct from lemma \ref{sc:mnbound}

\bigskip

\noindent $\Leftarrow$ \\
By using lemma \ref{re:norm} and apply it on the space the assumptions change to the following:
\begin{enumerate}[label=(\roman*)]
	\item $\A$ is closed and $\overline{D(\A)} = X$.
	\item $\mathbb{R}^*_+ \subset \rho(\A)$ and $\norm{R(\lambda, \A)} \le \frac{1}{\lambda}$ for all $\lambda > 0$.
\end{enumerate}
Which are just the conditions on Hille-Yosida for contractions and we get the desired result.
\end{proof}

\begin{theo}[Hille-Yosida for the Generale Case $(M, \omega)$] \label{thm:hy_mo}
	Let $\A: D(\A) \subset X \to X$ be a linear operator on a Banach space $X$. Then $\A$ is the IG of a $C_0$-SG $T(t)$ satisfying $\norm{T(t)} \le M\cdot e^{\omega t}$ for all $t \ge 0$ if and only if:
\begin{enumerate}[label=(\roman*)]
	\item $\A$ is closed and ($\overline{D(\A)} = X$).
	\item $\rho(\A) \supset (0, \infty)$ and for every $\lambda > 0$:
    $$
	    \norm{(\lambda^n R(\lambda, \A))^n} \le M, \quad \forall n \in \mathbb{N}
    $$
\end{enumerate}
\end{theo}

\begin{theo}
	Let $\A: D(\A) \subset X \to X$ be a UBLO and the IG of a $C_0$-SG $T(t)_{t \ge 0}$ then for all $x \in X$:
	$$
	T(t)x = \lim_{\lambda \to +\infty}{e^{t\A_{\lambda}}x}
	$$
	Where $\A_{\lambda}$ is the Yosida Approximation of $\A$.
\end{theo}

\begin{proof}
	This can have 2 cases depending on $\omega$.

	First case, let $\omega = 0$. Then by applying lemma \ref{re:norm} we get a contraction and it is direct by corollary \ref{ya:limit}

	\bigskip

	Second case, let $\omega > 0$. We want to get a bound on the approximation.
	\begin{align*}
		\norm{e^{t \A_{\lambda}}} &= \norm{e^{t (\lambda^2 R(\lambda, \A) - \lambda I)}} \\
					  &= e^{-\lambda t} \norm{e^{t \lambda^2 R(\lambda, \A)}} \\
					  &\le e^{-\lambda t} \sum_{k=0}^\infty \frac{(t \lambda^2)^k}{k!} \norm{R(\lambda, \A)^k} \\
					  &\le e^{-\lambda t} M \sum_{k=0}^\infty \frac{1}{k!} \left( \frac{t \lambda^2}{\lambda - \omega} \right)^k && \text{since } \norm{R(\lambda, \A)^k} \le \dfrac{M}{(\lambda - \omega)^k}\\
    &= M e^{-\lambda t} \exp\left( \frac{t \lambda^2}{\lambda - \omega} \right) \\
    &= M \exp\left( -\lambda t + \frac{t \lambda^2}{\lambda - \omega} \right)
\end{align*}
We can do the following calculation in the exponent:
\begin{align*}
	-\lambda + \frac{\lambda^2}{\lambda - \omega} &= \left( \frac{-\lambda(\lambda - \omega) + \lambda^2}{\lambda - \omega} \right) \\
						      & = \frac{\lambda \omega}{\lambda - \omega}\\
						      &= \omega \left( \frac{\lambda - \omega + \omega}{\lambda - \omega} \right)\\
						      &= \omega \left( 1 + \frac{\omega}{\lambda - \omega} \right)\\
						      &= \omega + \frac{\omega^2}{\lambda - \omega}
\end{align*}
Now for all $\lambda \ge 2 \omega$ we have $\lambda - \omega \ge \omega > 0$ and $\frac{1}{\lambda - \omega} \le \frac{1}{\omega}$, hence:
$$
\omega + \frac{\omega^2}{\lambda - \omega} \le \omega + \frac{\omega^2}{\omega} = 2 \omega
$$
And we get the bound:
\begin{equation} \label{hy:general_bound}
\norm{e^{t\A_{\lambda}}} \le M \cdot e^{2t\omega} \tag{*}
\end{equation}

Let $S(t) = e^{-\omega t}T(t)$, this $C_0$-SG is generated by $\A - \omega I$ (rescaling of $T(t)$).

Since $S(t)$ is $(M, 0)$ then by the first case we have:
$$
\forall{t} \ge 0, \, \forall{x} \in X \quad S(t)x = \lim_{\lambda \to +\infty}{e^{t(\A - \lambda I)_{\lambda}}}
$$

Equivalently:

$$
e^{- \omega t} T(t)x = \lim_{\lambda \to +\infty}{e^{t(\A - \lambda I)_{\lambda}}}
$$

Now we have to show that $\forall t \ge 0, \forall x \in X$ the following:

$$
e^{t[(\A - \omega I)_{\lambda} + \omega I]}x - e^{t \A_{\lambda}}x = 0
$$

We have the following:

\begin{align*}
	e^{t[(\A - \omega I)_{\lambda} + \omega I]}x - e^{t \A_{\lambda}}x &= e^{t[(\A - \omega I)_{\lambda} + \omega I - \A_{\lambda} + \A_{\lambda}]}x - e^{t \A_{\lambda}}x \\
									   &= e^{t (A_\lambda + H(\lambda))} x - e^{t A_\lambda} x && \text{by setting} \, H(\lambda) := (\A - \omega I)_{\lambda} + \omega I - \A_{\lambda} \\
    &= e^{t A_\lambda} e^{t H(\lambda)} x - e^{t A_\lambda} x \\
    &= e^{t A_\lambda} \left( e^{t H(\lambda)} - I \right) x  && \A_{\lambda} \, \text{and} \, H(\lambda) \, \text{commutes (why?)}
\end{align*}

\noindent Now, we aim to show that for any fixed $x \in D(A)$, $H(\lambda)x \to 0$ as $\lambda \to +\infty$.

By theorem \ref{hy:yosida_conv} we have:
\begin{align*}
    \lim_{\lambda \to \infty} H(\lambda)x &= \lim_{\lambda \to \infty} \left( (A - \omega I)_\lambda x + \omega x - A_\lambda x \right) \\
    &= (Ax - \omega x) + \omega x - Ax \\
    &= 0
\end{align*}
Thus, $H(\lambda)x \to 0$ for all $x \in D(A)$.

Now we have:
$$
e^{t[(\A - \omega I)_{\lambda} + \omega I]}x - e^{t \A_{\lambda}}x = e^{t A_\lambda} \left( e^{t H(\lambda)} - I \right) x
$$

By taking the norm:
$$
\norm{e^{t[(\A - \omega I)_{\lambda} + \omega I]}x - e^{t \A_{\lambda}}x} \le \norm{e^{t A_\lambda}} \norm{\left( e^{t H(\lambda)} - I \right) x}
$$

From \eqref{hy:general_bound}, we have that $\norm{e^{t A_\lambda}}$ is uniformaly bounded hence we need to work on $\norm{\left( e^{t H(\lambda)} - I \right) x}$.

Note that $e^{tH(\lambda)} = e^{t(B_\lambda + \omega I)} e^{-tA_\lambda}$. From the bound estimates shown earlier, both $\|e^{s(B_\lambda + \omega I)}\|$ and $\|e^{-sA_\lambda}\|$ are uniformly bounded for $s$ in a compact interval $[0, T]$ and sufficiently large $\lambda$. Let $C$ be this bound. Then:
\[ \| (e^{t H(\lambda)} - I) x \| \le t \cdot C \cdot \| H(\lambda)x \| \]
Since we proved that $\lim_{\lambda \to \infty} \|H(\lambda)x\| = 0$, it follows immediately that:
\[ \lim_{\lambda \to \infty} \| (e^{t H(\lambda)} - I) x \| = 0 \]
Finally, combining this with the uniform bound on $\|e^{t A_\lambda}\|$, we obtain:
\[ \lim_{\lambda \to \infty} \| e^{t (B_\lambda + \omega I)} x - e^{t A_\lambda} x \| = 0 \]
This proves that the limit is the same, regardless of the shift. This completed our proof.
\end{proof}

\section{Inverse Laplace Transform and Resolvent Formula}

We have established that for a $C_0$-SG $\left(T(t)\right)_{t \ge 0}$ with $\norm{T(t)} \le Me^{t\omega}$, the resolvent is given using the following formula:
$$
R(\lambda,\A)x = \int_0^\infty e^{-\lambda t} T(t)x \, dt \quad \text{for } \text{Re}(\lambda) > \omega.
$$
which is just the Laplace Tranfrom of the semigroup.

A major problem in semigroup theory is recovering $T(t)$ from $R(\lambda,\A)x$ (Inverting the Laplace Transform).

\begin{lemma}
	Let $X$ be BS and $\B \in \L(X)$ then $\forall \gamma > \norm{\B}$ we have:
	\begin{align*}
	e^{t\B} &= \frac{1}{2 \pi i} \int_{\gamma - i\infty}^{\gamma + i\infty} e^{\lambda t} R(\lambda, \B) d\lambda \\
		&= \frac{1}{2 \pi i} \int_{- \infty}^{+ \infty} e^{(\gamma + is)t} R(\gamma + is, \B) i ds
	\end{align*}
\end{lemma}

\begin{proof}
	To make sense and motivate this proof we start by showing that this is correct for the scalar case. Let $b \in \mathbb{C}$ and the exponential function $f(t) = e^{tb}$. The Laplace transform of $f(t)$ is:
	$$
\mathcal{L}\{e^{tb}\}(\lambda) = \int_0^\infty e^{-\lambda t} e^{tb} \, dt = \frac{1}{\lambda - b}, \quad \text{for } \Re(\lambda) > \Re(b)
	$$
	Now if we choose $\left|\lambda\right| > \left|b\right|$. We can expand $(\lambda - b)^{-1}$ as a convergent geometric series:
	Now if we choose $\gamma$ such that $\Re(\gamma) > \left| b \right|$. Let show that:
	$$
		e^{tb} = \frac{1}{2\pi i} \int_{\gamma - i\infty}^{\gamma + i\infty} e^{\lambda t} \frac{1}{\lambda - b} \, d\lambda
	$$
	For a circle $C(0, r)$ with radius $r > |b|$, we can expand the term $(\lambda - b)^{-1}$ as a geometric series for $|\lambda| > |b|$:
	$$
	\frac{1}{\lambda - b} = \frac{1}{\lambda(1 - \frac{b}{\lambda})} = \frac{1}{\lambda} \sum_{k = 0}^{\infty}{\frac{b^k}{\lambda^k}} = \sum_{k = 0}^{\infty}{\frac{b^k}{\lambda^{k+1}}}
	$$
	Plugging this into the formula above we get:
	\begin{align*}
		\frac{1}{2\pi i} \int_{\gamma - i\infty}^{\gamma + i\infty} e^{\lambda t} \frac{1}{\lambda - b} \, d\lambda
		&= \frac{1}{2\pi i} \int_{\gamma - i\infty}^{\gamma + i\infty} e^{\lambda t} \sum_{k = 0}^{\infty}{\frac{b^k}{\lambda^{k+1}}} \, d\lambda && (\text{the choice of} \, \gamma \, \text{allow for this}) \\
		&= \frac{1}{2\pi i} \sum_{k = 0}^{\infty} b^k \int_{\gamma - i\infty}^{\gamma + i\infty} \frac{e^{\lambda t}}{\lambda^{k+1}} \, d\lambda && \left(\text{The sum is uniformly convergent}\right) \\
		&= \frac{1}{2\pi i} \sum_{k = 0}^{\infty} 2\pi i \frac{b^k \, t^k}{k!} && \left(\text{Cauchy Integral Formula}\right) \\
		&= e^{bt}
	\end{align*}
	This shows that the formula works for the scalar case.

	\bigskip

	The case of the operator follows the same steps and same resoning.

	Again let $C(0,r)$ with radius $r$ such that $r > \norm{\B}$ For any $\lambda \in C(0,r)$, we have $|\lambda| > \norm{\B}$, which implies $\|\frac{\B}{\lambda}\| < 1$.

In this region, the resolvent $R(\lambda, \B) = (\lambda I - \B)^{-1}$ admits a uniformly convergent Neumann series expansion:
$$
R(\lambda, \B) = \frac{1}{\lambda} \left( I - \frac{\B}{\lambda} \right)^{-1} = \sum_{k=0}^\infty \frac{\B^k}{\lambda^{k+1}}.
$$
We substitute this series into the contour integral over $C(0, r)$:
\begin{align*}
\frac{1}{2\pi i} \int_{C(0,r)} e^{\lambda t} R(\lambda,\B) \, d\lambda 
&= \frac{1}{2\pi i} \int_{C{0,r}} e^{\lambda t} \left( \sum_{k=0}^\infty \frac{\B^k}{\lambda^{k+1}} \right) d\lambda \\
&= \frac{1}{2\pi i} \sum_{k=0}^\infty \B^k \left[ \int_{C_r} \frac{e^{\lambda t}}{\lambda^{k+1}} \, d\lambda \right] \\
&= \frac{1}{2\pi i} \sum_{k=0}^\infty \B^k \left[2\pi i \frac{t^k}{k!} \right] \\
&= \sum_{k=0}^\infty \frac{t^k \B^k}{k!} \\
&= e^{t\B}
\end{align*}
\end{proof}

\section{Spectral Mapping Theorem}
Let $\left( T(t) \right)_{t \ge 0}$ be $C_0$-SG with $\A$ as its IG, then what is the relation betweem $\sigma(T(t))$ and $\sigma(\A)$ ?

\begin{exercise}{}{}
	Let $\left( T(t) \right)_{t \ge 0}$ be UC-SG with $\A$ as its IG then $e^{t\sigma(\A)} = \sigma(T(t))$.
\end{exercise}

\begin{exercise}{}{}
	Take $X = \left\{f:[0,1] \to \mathbb{C} \mid f \, \text{is continuous and } f(1) = 0 \right\}$ then ($X, \norm{\cdot}_{\infty})$ is BS. Define:
	$$
	T(t)f(x) = \begin{cases*}
		\, $f(x + t)$ & $x + t \le 1$ \\
		\, $0$ & $x + t > 1$
                   \end{cases*}
	$$
	Then:
	\begin{enumerate}
		\item $\left(T(t)\right)_{t \ge 0}$ is a $C_0$-SG.
		\item $\A$ the IG and $\A f = f^{'}$. Determine $D(\A)$.
		\item Check $\forall \lambda \in \mathbb{C}$, $\exists ! f \in X$ such that $(\lambda I - \A)f = g$ for all $g \in X$.
		\item $\sigma(\A) = \emptyset$.
		\item $\sigma(T(t)) \neq \emptyset$.
	\end{enumerate}
\end{exercise}

\begin{lemma}\label{spec:prep}[Preparatory Lemma]
	Let $(T(t))_{t \ge 0}$ be a $C_0$-SG and let $\A$ be its IG. For all $t \ge 0$ and for all $x \in X$ define:
	$$
	\B_{\lambda}(t)x = \int\limits_{0}^{t} e^{\lambda(t-s)}T(s) \, x \, ds
	$$
	then $(\lambda I - \A)\B_{\lambda}(t)x = e^{\lambda t}x - T(t)x$.
\end{lemma}

\begin{proof}
	It is enough to show that:
	$$
	\A \B_{\lambda}(t)x = T(t)x  + \lambda \B_{\lambda}(t)x - e^{\lambda t}x
	$$
	Recall that:
	$$
	\A = \lim\limits_{h \to 0^{+}} \frac{T(h) - Id}{h}
	$$
	Then we have:
	\begingroup
	\allowdisplaybreaks
	\begin{align*}
		\frac{T(h) - Id}{h} B_{\lambda}(t)x &= \frac{1}{h} \int\limits_{0}^{t} e^{\lambda(t-s)}\left(T(h)T(s) - T(s)\right) \, x \, ds \\
						    &= \frac{1}{h} \left[\int\limits_{0}^{t} e^{\lambda(t-s)} T(s + h)\, x \, ds - \int\limits_{0}^{t} e^{\lambda(t-s)} T(s) \, x \, ds\right] \\
						    &= \frac{1}{h} \left[\int\limits_{h}^{t + h} e^{\lambda(t-s + h)} T(s)\, x \, ds - \int\limits_{0}^{t} e^{\lambda(t-s)} T(s) \, x \, ds\right] \\
						    &= \frac{1}{h} \left[e^{\lambda h}\int\limits_{h}^{t + h} e^{\lambda(t - s)} T(s)\, x \, ds - \int\limits_{0}^{t} e^{\lambda(t-s)} T(s) \, x \, ds\right] \\
						    &= \frac{1}{h} \left[\left(e^{\lambda h} - 1\right)\int\limits_{h}^{t + h} e^{\lambda(t - s)} T(s)\, x \, ds + \int\limits_{h}^{t + h} e^{\lambda(t - s)} T(s)\, x \, ds - \int\limits_{0}^{t} e^{\lambda(t-s)} T(s) \, x \, ds\right] \\
						    &= \frac{1}{h} \left[\left(e^{\lambda h} - 1\right)\int\limits_{h}^{t + h} e^{\lambda(t - s)} T(s)\, x \, ds + \int\limits_{t}^{t + h} e^{\lambda(t - s)} T(s)\, x \, ds - \int\limits_{0}^{h} e^{\lambda(t-s)} T(s) \, x \, ds\right]
	\end{align*}
	Now taking $\lim\limits_{h \to 0^{+}}$ we get:
	$$
	\A \B_{\lambda}(t)x = \lambda \B_{\lambda}(t)x + T(t)x - e^{\lambda t}x
	$$
	\endgroup
\end{proof}

\begin{remark}
	In the lemma above $\A$ commutes if $x \in D(\A)$, and we get:
	$$
\B_{\lambda}(t)(\lambda I - \A)x = (\lambda I - \A)\B_{\lambda}(t)x = e^{\lambda t}x - T(t)x
	$$
	This is true because of the fact that $\A$ is the limit.
\end{remark}

\begin{theo}
	Let $\left(T(t)\right)_{t \ge 0}$ be a $C_0$-SG generated by $\A$. Then for all $t \ge 0$:
	$$
		e^{t \sigma(A)} \subset \sigma(T(t))
	$$
\end{theo}

\begin{proof}
We prove the equivalent statement: if $e^{\lambda t} \in \rho(T(t))$, then $\lambda \in \rho(\A)$.

Assume $e^{\lambda t} \in \rho(T(t))$. Then the operator $(e^{\lambda t} I - T(t))$ is invertible with a bounded inverse $Q = (e^{\lambda t} I - T(t))^{-1} \in \mathcal{L}(X)$.

From Lemma \ref{spec:prep}, we have:
$$
(\lambda I - \A) \B_\lambda(t) = e^{\lambda t} I - T(t).
$$

\begin{claim}
	$Q$ and $\B_{\lambda}(t)$ commute. (Prove it)
\end{claim}
Multiplying by $Q$ on the right:
$$
(\lambda I - \A) \B_\lambda(t) Q = I.
$$
Similarly, for $x \in D(\A)$:
$$
Q \B_\lambda(t) (\lambda I - \A) x = x.
$$
Defining $R(\lambda) = \B_\lambda(t) Q$, we see that $R(\lambda)$ acts as a two-sided inverse for $(\lambda I - A)$. Since $B_\lambda(t)$ and $Q$ are bounded, $R(\lambda)$ is bounded. Thus $\lambda \in \rho(\A)$.

By contraposition, $\lambda \in \sigma(\A) \implies e^{\lambda t} \in \sigma(T(t))$.
\end{proof}

\pagebreak

While the full spectral mapping theorem fails, a precise relationship holds for the point spectrum $\sigma_p$ and the residual spectrum $\sigma_r$.

Recall the decomposition of the spectrum:
$$
\sigma(\A) = \sigma_p(\A) \cup \sigma_c(\A) \cup \sigma_r(\A)
$$
where $\sigma_p$ denotes point spectrum (eigenvalues), $\sigma_c$ continuous spectrum, and $\sigma_r$ residual spectrum.

\begin{theo}
	Let $\left(T(t)\right)_{t \ge 0}$ be a $C_0$-SG generated by $\A$. Then:
$$
e^{t \sigma_p(\A)} \subseteq \sigma_p(T(t)) \subseteq e^{t \sigma_p(\A)} \cup \left\{0\right\}
$$
Precisely, if $e^{\lambda t} \in \sigma_p\left(T(t)\right)$, then there exists $k \in \mathbb{Z}$ such that:
$$
\lambda_k = \lambda + \frac{2\pi i k}{t} \in \sigma_p(\A)
$$
\end{theo}

\begin{proof}
First inclusion, $\lambda \in \sigma_p(\A) \implies e^{\lambda t} \in \sigma_p(T(t))$.

Let $\lambda \in \sigma_p(\A)$. Then there exists $x_0 \neq 0$ such that $\A x_0 = \lambda x_0$ and by lemma \ref{spec:prep} we have:
$$
\B_{\lambda}(t) (\lambda I - \A) x = \left(e^{\lambda t} I - T(t)\right)x \quad\quad \forall x \in D(\A)
$$
Now take $x = x_0$ and note the fact that $(\lambda I - \A) x_0 = 0$:
$$
\left(e^{\lambda t} I - T(t)\right)x_0 = 0
$$
Hence $\ker{\left(e^{\lambda t} I - T(t)\right)} \neq \left\{ 0 \right\}$ then $e^{\lambda t} \in \sigma_p(T(t))$.

For the second inclusion, let $\mu \in \sigma_p(T(t))$ with $\mu \neq 0$. Write $\mu = e^{\lambda t}$ for some $\lambda \in \mathbb{C}$.

There exists $x_0 \neq 0$ such that $T(t)x_0 = e^{\lambda t}x_0$.

Consider the function $\phi_{x_0}(s)$ defined by:
$$
\phi_{x_0}(s) = e^{-\lambda s} T(s) x_0
$$
Clearly:
$$
\phi_{x_0}(0) = x_0 = \phi_{x_0}(t)
$$
We check if $\phi_{x_0}$ is periodic with period $t$.
$$
\phi_{x_0}(s+t) = e^{-\lambda(s+t)} T(s+t)x_0 = e^{-\lambda s} e^{-\lambda t} T(s) T(t) x_0
$$
Using the eigen-property $T(t)x_0 = e^{\lambda t}x_0$:
$$
\phi_{x_0}(s+t) = e^{-\lambda s} e^{-\lambda t} T(s) (e^{\lambda t} x_0) = e^{-\lambda s} T(s) x_0 = \phi_{x_0}(s)
$$
Since $\phi_{x_0}(s)$ is a $t$-periodic continuous function (and assuming $x_0 \in X$), we can expand it into a Fourier series. The $k$-th Fourier coefficient is:
$$
\hat{x}_k = \frac{1}{t} \int_0^t e^{-2\pi i k s / t} \phi_{x_0}(s) \, ds
$$
Substituting $\phi_{x_0}(s) = e^{-\lambda s} T(s) x_0$:
$$
\hat{x}_k = \frac{1}{t} \int_0^t e^{-(\lambda + \frac{2\pi i k}{t}) s} T(s) x_0 \, ds
$$
Let $\lambda_k = \lambda + \frac{2\pi i k}{t}$. Since $\phi_{x_0}$ is not identically zero (as $\phi_{x_0}(0) = x_0 \neq 0$), at least one coefficient $\hat{x}_k$ must be non-zero.

We will show that if $\hat{x}_k \neq 0$, then $\lambda_k \in \sigma_p(\A)$.

We use the resolvent formula. For $\Re(\gamma)$ sufficiently large, $R(\gamma, \A) = \int_0^\infty e^{-\gamma s} T(s) ds$. Applying this to $x_0$:
$$
R(\gamma, \A)x_0 = \int_0^\infty e^{-\gamma s} T(s) x_0 \, ds
$$
Decompose the integral over intervals $[nt, (n+1)t]$:
\begin{align*}
R(\gamma, \A)x_0 &= \sum_{n=0}^\infty \int_{nt}^{(n+1)t} e^{-\gamma s} T(s) x_0 \, ds \\
&= \sum_{n=0}^\infty \int_0^t e^{-\gamma(nt + \tau)} T(nt + \tau) x_0 \, d\tau \quad (\text{let } s = nt + \tau)
\end{align*}
Using the semigroup property $T(nt+\tau)x_0 = T(\tau)T(t)^n x_0 = T(\tau) e^{n \lambda t} x_0$:
\begin{align*}
R(\gamma, \A)x_0 &= \sum_{n=0}^\infty e^{-n \gamma t} e^{n \lambda t} \int_0^t e^{-\gamma \tau} T(\tau) x_0 \, d\tau \\
&= \left( \sum_{n=0}^\infty e^{-n(\gamma - \lambda)t} \right) \int_0^t e^{-\gamma \tau} T(\tau) x_0 \, d\tau
\end{align*}
The geometric series converges if $\Re(\gamma) > \Re(\lambda)$:
$$
\sum_{n=0}^\infty (e^{-(\gamma - \lambda)t})^n = \frac{1}{1 - e^{-(\gamma - \lambda)t}}
$$
Thus:
$$
R(\gamma, \A)x_0 = \frac{1}{1 - e^{(\lambda - \gamma)t}} \int_0^t e^{-\gamma \tau} T(\tau) x_0 \, d\tau
$$
The function $\gamma \mapsto R(\gamma, \A)x_0$ is meromorphic. The poles of the resolvent indicate the spectrum. The denominator vanishes when:
$$
e^{(\lambda - \gamma)t} = 1 \iff (\lambda - \gamma)t = 2\pi i k \iff \gamma = \lambda - \frac{2\pi i k}{t}
$$
Let $\mu_k = \lambda + \frac{2\pi i k}{t}$. The resolvent has a pole at $\mu_k$.
Specifically, the residue near the pole relates to the existence of an eigenvector. If we analyze the limit:
$$
\lim_{\gamma \to \lambda_k} (\gamma - \lambda_k) R(\gamma, \A) x_0 \neq 0 \implies \lambda_k \in \sigma_p(\A).
$$
This confirms that the eigenvalues of $\A$ are exactly the logarithms of the eigenvalues of $T(t)$ (modulo $2\pi i / t$).
\end{proof}

\chapter{Applications to Partial Differential Equations}

Let $X$ be a Banach space and $A: D(A) \subset X \rightarrow X$ be a linear operator. We define the homogeneous and inhomogeneous abstract Cauchy problems as follows:

\begin{itemize}
    \item Homogeneous Cauchy Problem $h-(CP)_{0, x_0}$: 
    $$ \frac{du}{dt} = Au, \quad t \ge 0 $$ 
    $$ u(0) = x_0 $$ 
    
    \item Inhomogeneous Cauchy Problem $inh-(CP)_{0,x_0}$: 
    $$ \frac{du}{dt} = Au + f(t), \quad t \ge 0 $$ 
    $$ u(0) = x_0 $$ 
    where $f: \mathbb{R}_+ \rightarrow X$. 
\end{itemize}

\section{The Cauchy Problem}

\begin{definition}
We distinguish three types of solutions for the Cauchy problem:
\begin{enumerate}
    \item \textbf{Strong Solution:} A function $x(\cdot) \in C^1(\mathbb{R}_+, D(\A))$ such that for all $t \ge 0$, $x(t) \in D(\A)$ and $\frac{dx}{dt} = \A x(t)$. 
    
    \item \textbf{Mild Solution:} A function $x(\cdot) \in C^0\left(\mathbb{R}_+, D(\A)\right) \cap C^1(\mathbb{R}_+^*, D(\A))$ with $x(0) = x_0$, and for all $t > 0$, $x(t) \in D(\A)$ such that $\frac{dx}{dt} = \A x(t)$. 
    
    \item \textbf{Weak Solution:} A function $x(\cdot) \in C^0(\mathbb{R}_+, X)$ such that $x(0) = x_0$, and for all $x^* \in D(A^*)$, the map $t \mapsto \langle x^*, x(t) \rangle$ is $C^1$ and satisfies:
    $$
    \frac{d}{dt} \langle x^*, x(t) \rangle = \langle A^* x^*, x(t) \rangle
    $$
\end{enumerate}
\end{definition}

\begin{remark} Some properties of the solutions and constraints on the initial value:
\begin{enumerate}
    \item If $x_0 \notin D(A)$, there is no valid strong solution for $(CP)_{0,x_0}$. 
    \item For all $x_0 \in X$, $t \mapsto T(t)x_0$ is the unique solution of $(CP)_{0,x_0}$. 
\end{enumerate}
\end{remark}

\begin{example}
Consider the transport equation:
$$
u_t = u_x \quad\quad x \in [0,1], \; t \ge 0
$$
subject to boundary conditions at $t=0$, $u(0, \cdot) = u_0(\cdot) \in X$.

We associate this problem with the operator $\A$ defined by $\A u = \frac{du}{dx}$. To define the weak solution, we need to determine the action of the adjoint operator $\A^*$.

Let $\varphi$ be a test function (of $D(A^*)$). We compute the pairing $\langle \A u, \varphi \rangle$ using the inner product in $L^2([0,1])$:

$$ \langle \A u, \varphi \rangle = \int_0^1 (\A u)(x) \varphi(x) \, dx = \int_0^1 \frac{du}{dx}(x) \varphi(x) \, dx $$

By integration by parts:
$$ \int_0^1 u'(x) \varphi(x) \, dx = \left[ u(x)\varphi(x) \right]_0^1 - \int_0^1 u(x) \varphi'(x) \, dx $$

Using the Dirichlet boundary conditions ($u(t,1) = u(t,0) = 0$), the boundary term $\left[ u(x)\varphi(x) \right]_0^1$ vanishes. Thus, we obtain:
$$ \langle \A u, \varphi \rangle = - \int_0^1 u(x) \varphi'(x) \, dx = \int_0^1 u(x) \left( -\frac{d\varphi}{dx} \right) \, dx $$

This can be rewritten in terms of the inner product as:
$$ \langle \A u, \varphi \rangle = \langle u, -\varphi' \rangle $$

From the definition of the adjoint operator $\langle \A u, \varphi \rangle = \langle u, \A^* \varphi \rangle$, we identify:
$$ \A^* \varphi = -\frac{d\varphi}{dx} $$

Therefore, the condition for $u(t)$ to be a weak solution:
$$ \frac{d}{dt} \langle \varphi, u(t) \rangle = \langle A^* \varphi, u(t) \rangle $$
becomes:
$$ \frac{d}{dt} \int_0^1 u(t,x) \varphi(x) \, dx = -\int_0^1 u(t,x) \varphi'(x) \, dx $$
\end{example}

\pagebreak

\section{Homogeneous Cauchy Problem}

\begin{theo} \label{app:sol}
Let $\A$ be the IG of a $C_0$-SG $(T(t))_{t \ge 0}$, then: 
\begin{enumerate}
	\item \label{app:stg} For all $x_0 \in D(\A)$, there exists a unique strong solution of $(CP)_{0,x_0}$ given by $u(t) = T(t)x_0$, which is also a mild and weak solution. 
	\item \label{app:weak} For all $x_0 \in X$, $t \mapsto T(t)x_0$ is a weak solution of $(CP)_{0,x_0}$. 
\end{enumerate}
\end{theo}

\begin{proof} The proof of part \ref{app:stg} is direct.

\bigskip

For part \ref{app:weak}: Let $x_0 \in X$ and choose a sequence $x_n \in D(\A)$ such that $x_n \rightarrow x_0$ (why is this possible?). For all $x^* \in D(\A^*)$ and $t \ge 0$, the function $t \mapsto \langle x^*, T(t)x_n \rangle$ is $C^1$, and we have:
$$
\frac{d}{dt} \langle x^*, T(t)x_n \rangle = \langle x^*, \A T(t)x_n \rangle = \langle \A^* x^*, T(t)x_n \rangle
$$
Integrating yields:
$$ \langle x^*, T(t)x_n \rangle = \langle x^*, x_n \rangle + \int_0^t \langle A^* x^*, T(s)x_n \rangle ds $$ 
By passing to the limit as $n \rightarrow \infty$ (show how), we extend this formula for all $x_0 \in X$, showing $t \mapsto T(t)x_0$ is a weak solution. 

\bigskip

Now to show the uniqueness of the solution, consider two weak solutions $x_1(\cdot)$ and $x_2(\cdot)$ of $(CP)_{0,x_0}$. 
Set $u = x_1 - x_2$. 
Then $u$ is a weak solution of the problem with initial condition $0$. 
For all $t \ge 0$ and $x^* \in D(A^*)$, we have:
$$
\langle x^*, u(t) \rangle = \int_0^t \langle A^* x^*, u(s) \rangle ds
$$

Let $U(t) = \int_0^t u(s)ds$, with $U(0) = 0$. 
Then, $\langle x^*, u(t) \rangle = \langle A^* x^*, U(t) \rangle$, which implies:
\begin{equation}\label{sol:weak:uniq}
\frac{d}{dt} \langle x^*, U(t) \rangle = \langle A^* x^*, U(t) \rangle \tag{***}
\end{equation}

\begin{claim}
	If $\left( T(t) \right)_{t \ge 0}$ is a $C_0$-SG with $\A$ as its IG then $T(t)^* D(\A^*) \subset D(\A^*)$ and $T(t)^*\, \text{and } \A^*$ commute, that is:
	$$
		T(t)^* \A^* = \A^* T(t)^*
	$$
\end{claim}

\begin{proof}[Proof of claim]
We have to show that $\forall{x^*} \in D(\A^*), \forall{x} \in D(\A), \forall{t} \ge 0$, the follwing:
$$
\exists{\xi^*} \in X^*, \langle T(t)^* x^*, \A x \rangle = \langle \xi^*, x \rangle 
$$
We have the following:
\begin{align*}
	\langle T(t)^* x^*, \A x \rangle 
	&= \langle x^*, T(t) \A x \rangle \\
	&= \langle x^*, \A T(t) x \rangle \\
	&= \langle \A^* x^*, T(t) x \rangle \\
	&= \langle T(t)^* \A^* x^*, x \rangle
\end{align*}
Then $T(t)^* \A^* x^* \in D(\A^*)\, \text{and } T(t)^* \A^* = \A^* T(t)^*$.
\end{proof}

Apply \eqref{sol:weak:uniq} replacing $x^*$ with $T^*(t^* - t)x^*$, where $0 \le t \le t^*$: 
$$ \frac{d}{dt} \langle T^*(t^* - t)x^*, U(t) \rangle = \langle -\A^* T^*(t^* - t)x^*, U(t) \rangle + \langle T^*(t^* - t)x^*, u(t) \rangle $$ 
Since $\langle T^*(t^* - t)x^*, u(t) \rangle = \langle \A^* T^*(t^* - t)x^*, U(t) \rangle$, the derivative is $0$. 

Integrating from $t=0$ to $t=t^*$ yields:
$$
\langle x^*, U(t^*) \rangle - \langle T^*(t^*)x^*, U(0) \rangle = 0
$$
Since $U(0) = 0$, we get $\langle x^*, U(t^*) \rangle = 0$ for all $t^* \ge 0$. Since this holds for all $x^* \in D(\A^*)$ and $D(\A^*)$ is weak*-dense, $U(t) \equiv 0$.

Differentiating gives $u(t) = 0$, proving uniqueness. 
\end{proof}

\begin{theo} \label{sol:mild}
	Let $\A: D(\A) \subset X \rightarrow X$ be a closed, densely defined operator. If there exists $\omega$ such that for $\lambda > \omega$, $\lambda \in \rho(\A)$ and $\norm{R(\lambda, \A)} = o(e^{\sigma \lambda})$, 
then $(CP)_{0,x_0}$ admits a unique mild solution. 
\end{theo}

We first prove a necessary lemma for uniqueness:
\begin{lemma} \label{sol:mild:uniq:lemma}
Let $u: [0, T] \rightarrow X$ be continuous with $u(0)=0$. Assume there exists $M \ge 0$ such that for all $n \ge 0$,
$$
\norm{\int_0^T e^{ns} u(s) ds} \le M
$$
Then $u(t) = 0$ for all $t \in [0, T]$. 
\end{lemma}

\begin{proof}
	For any $x^* \in X^*$, define the continuous function $\varphi(s) := \langle x^*, u(s) \rangle$. By assumption, $\norm{\int_0^T e^{ns} \varphi(s) ds} \le M$ for all $n$. 
Define the sequence of functions:
$$ \psi_n(t) = \sum_{k \ge 1} \frac{(-1)^{k+1}}{k!} e^{knt} = 1 - e^{-e^{nt}} $$ 
We consider the integral:
$$ \int_0^T \psi_n(t - T + s) \varphi(s) ds $$ 
By substituting the sum definition of $\psi_n$ and using the bound on $\varphi$, we have:
\begin{align*}
\norm{\int_0^T \sum_{k \ge 1} \frac{(-1)^{k+1}}{k!} e^{kn(t - T + s)} \varphi(s) ds}
&\le \sum_{k \ge 1} \frac{1}{k!} e^{kn(t - T)} \norm{\int_0^T e^{kns} \varphi(s) ds} \\
&\le M \sum_{k \ge 1} \frac{(e^{n(t-T)})^k}{k!}\\
&= M (e^{e^{n(t-T)}} - 1)
\end{align*}

For $t < T$, as $n \rightarrow \infty$, $n(t-T) \rightarrow -\infty$, thus $e^{n(t-T)} \rightarrow 0$, meaning the integral evaluates to $0$. 

On the other hand, analyzing the behavior of $\psi_n(t - T + s)$ directly as $n \rightarrow \infty$:
\begin{itemize}
    \item If $s > T - t$, then $t - T + s > 0$, so $\psi_n \rightarrow 1$. 
    \item If $s < T - t$, then $t - T + s < 0$, so $\psi_n \rightarrow 0$. 
\end{itemize}
By the Lebesgue Dominated Convergence Theorem, 
$$ \lim_{n \rightarrow \infty} \int_0^T \psi_n(t - T + s) \varphi(s) ds = \int_{T-t}^T \varphi(s) ds $$ 
Equating the two limits, we deduce $\int_{T-t}^T \varphi(s) ds = 0$ for all $t$. 
Differentiating with respect to $t$ gives $\varphi(T-t) = 0$, implying $\varphi \equiv 0$, and thus $u \equiv 0$. 
\end{proof}

\begin{proof}[Proof of theorem \ref{sol:mild}]
	Without loss of generality we can assume $\omega = 0$ since if it is not then the transform $\A \mapsto \A - zI$, shifts the coefficient by $e^{zt}$, and:
	$$
	\frac{du_z}{dt} = (\A +zI)u_z
	$$
	$$
	\frac{du}{dt} = \A u
	$$


Assume $u$ is a mild solution of $(CP)_{0,0}$, meaning $u(0)=0$ and $\frac{du}{dt} = \A u$. For $\lambda > 0$, define $v(t) = R(\lambda, \A)u(t)$. Since
$$
\frac{d}{dt} R(\lambda, \A)u(t) = R(\lambda, \A) \A u(t) = R(\lambda, \A)(\A - \lambda I + \lambda I)u(t) = -u(t) + \lambda R(\lambda, \A)u(t)$$
we get $\frac{dv}{dt} - \lambda v(t) = -u(t)$. 
Solving this ODE yields:
$$
R(\lambda, \A)u(t) = \int_0^t e^{\lambda(t-s)} u(s) ds
$$
Multiplying by $e^{- \sigma \lambda}$, we get:
$$
e^{- \sigma \lambda}R(\lambda, \A)u(t) = \int_0^t e^{\lambda(t - \sigma - s)} u(s) ds
$$

\begin{claim}
	$\lim\limits_{\lambda \to +\infty} \int_0^t e^{\lambda(t - \sigma - s)} u(s) ds = 0$.
\end{claim}

\begin{proof}
	Do it. \textcolor{blue}{(Hint: use the bond from the assumption and lemma \ref{sol:mild:uniq:lemma})}
\end{proof}

By the claim we get $u \equiv 0$.
\end{proof}

\begin{exercise}{Homework}{}
Let $X$ be a Hilbert Space and $\A$ be defined by 
$$
\A y = \sum \lambda_n \langle y, e_n \rangle e_n
$$
over an orthonormal basis $\{e_n\}$, with real numbers $\lambda_n \nearrow +\infty$. Its domain is ${D(\A) = \{ y \in X \mid \sum \lambda_n^2 \langle y, e_n \rangle^2 < +\infty \}}$. 
\begin{enumerate}
    \item Prove $\A$ is self-adjoint.
    \item Prove it generates a $C_0$-SG.
\end{enumerate}
\textcolor{blue}{(Hint: $\lambda I - \A$ is formally bounded if and only if $\lim\limits_{n \ge 1} \left| \lambda - \lambda_n \right| > 0$ )}
\end{exercise}

What can we say without $\A$ being the IG of some $C_0$-SG?

\begin{theo}[Existence of Solution]
Let $\A: D(\A) \subset X \to X$ be a UBLO with $\overline{D(\A)} = X$ such that $\rho(\A) \supset (\lambda_0, \infty)$, $\lambda_0 \in \mathbb{R}$ and
$$
\lim\limits_{\lambda \to \infty} \frac{1}{\lambda} \ln \norm{R(\lambda, \A)} = 0
$$
then $\forall x \in X \; (CP)_{0, x}$ admits a mild solution.
\end{theo}

Note the difference between the assumption here and Hille-Yosida assumption, so one can not use Hille-Yosida directly.

\begin{lemma} \label{app:sol:dense}
	Let $\A: D(\A) \supset X \to X$ be a UBLO then:
	\begin{enumerate}
		\item $\forall{\lambda} \in \rho(\A)\, D(\A^2) = \left(\lambda I - \A\right)^{-1} D(\A)$.
		\item If $\rho(\A) \neq \emptyset$ and $\overline{D(\A)} = X$ then $\overline{D(\A^2)} = X$.
	\end{enumerate}
\end{lemma}

\begin{proof}
	Left as an exercise.
\end{proof}

\begin{theo}[Uniqueness of Solution]
	Let $\A: D(\A)\subset X \to X$ be a UBLO with ${\overline{D(\A)} = X}$ and $\rho(\A) \neq \emptyset$. $\forall x \in D(\A)$, $(CP)_{0,x}$ admits a mild solution if and only if $\A$ is the IG of some $C_0$-SG.
\end{theo}

\begin{proof}
	Let's start with ($\impliedby$). Assume $\A$ is the IG of some $C_0$-SG then by theorem \ref{app:sol} part \ref{app:stg} we have a unique strong solution which is also mild.

	\bigskip

	For ($\implies$). We need to construct a $C_0$-SG with $\A$ as it IG. For this we use the flow of the solution. For all $x \in D(\A)$ we have:
	\begin{align*}
		\frac{du}{dt} &= \A u \\
		u(0) &= x
	\end{align*}
	Since $u$ is the unique mild solutions of $(CP)_{0, x}$ (actually it is a strong solution as well, why?). Then the mapping:
	\begin{align*}
		u: \mathbb{R}_+ \times D(\A) &\to D(\A) \\
		(t, x) &\mapsto u(t, x)
	\end{align*}
	is well degined. We can now consider the flow associated with $(CP)_{0,x}$. For each $t \ge 0$, define the operator $T(t): D(\A) \to D(\A)$ by:
	$$
		T(t)x = u(t,x)
	$$
	We must prove that the family $\left(T(t)\right)_{t \ge 0}$ forms a $C_0$-SG.
	\begin{enumerate}
	    \item Identity: $T(0)x = u(0,x) = x$. Thus, $T(0) = Id_{D(\A)}$.
	    \item Linearity: The operator $T(t)$ is linear due to the linearity of the derivative and the operator $\A$.
	    \item Semigroup Property: We need to show $T(t+s) = T(t)T(s)$.
	\end{enumerate}

	    By definition, $T(t)T(s)x = u(t, T(s)x)$, which is the solution evaluated at time $t$ for the initial value problem starting at $T(s)x$. Similarly, $T(t+s)x = u(t+s, x)$, which is the solution evaluated at time $t+s$ for the initial value problem starting at $x$. Due to the time-invariance of the equation $\frac{du}{dt} = Au$ and the uniqueness of the mild solution, advancing the flow by $s$ and then by $t$ is equivalent to advancing it by $t+s$. Hence, $T(t+s) = T(t)T(s)$.

	    \bigskip

	    Now we need to find the bound on $T(t)$ to show it is $C_0$-SG. To do this we need to equip the domain $D(\A)$ with a suitable Banach space structure (put a complete norm on it).

	We put the graph norm on $D(\A)$:
$$
|x|_G = \norm{x} + \norm{\A x}
$$
With this and the fact that $\A$ is closed (why?), $(D(\A), |\cdot|_G)$ is a Banach space. This will be denoted by $\left[D(\A)\right]$.

Let $t_0 > 0$, define $X_{t_0}$ as follows:
$$
X_{t_0} = C^0([0, t_0], [D(\A)])
$$
Clearly, $(X_{t_0}, \norm{\cdot}_{\infty})$ is a Banach space.

Define the operator $S_{t_0}$ by:
\begin{align*}
	S_{t_0}: [D(A)] &\to X_{t_0} \\
	x &\mapsto (t \mapsto u(t,x) = T(t)x)
\end{align*}
$S_{t_0}$ is well-defined and linear because of the uniqueness of solutions of $(CP)_{0,x}$.

\begin{claim}
    The operator $S_{t_0}$ is closed.
\end{claim}

\begin{proof}
    Suppose $x_n \to x$ in $[D(\A)]$ and $S_{t_0}(x_n) \to v$ in $X_{t_0}$. We aim to show that $S_{t_0}(x) = v$.

    By definition, $S_{t_0}(x_n) = u(\cdot, x_n)$. The convergence in $X_{t_0}$ implies:
    $$ ||S_{t_0}(x_n) - v||_{\infty} = \sup_{t \in [0, t_0]} |u(t, x_n) - v(t)|_G \to 0 \quad \text{as } n \to \infty $$
    Because the map is continuous, and $u$ is the unique solution, we must have $v(t) = u(t,x)$. Therefore, $S_{t_0}(x) = v$.
\end{proof}

Because $S_{t_0}$ is a closed operator acting between two Banach spaces, the Closed Graph Theorem implies that $S_{t_0}$ is bounded. Thus, there exists a constant $C_{t_0} > 0$ such that for all $x \in [D(\A)]$:
$$ \norm{S_{t_0}(x)}_{\infty} \le C_{t_0} |x|_G $$

Consequently, for all $t \in [0, t_0]$, we have $|T(t)x|_G \le C_{t_0} |x|_G$. This demonstrates that ${T(t) \in \mathcal{L}([D(\A)])}$, meaning $\left(T(t)\right)_{t \ge 0}$ is a $C_0$-SG on $[D(\A)]$.

Now, we prove that the semigroup commutes with its generator on appropriate domains.

\begin{claim}
    For all $y \in D(\A^2)$, $T(t)\A y = \A T(t)y$.
\end{claim}

\begin{proof}
    Let $y \in D(\A^2)$. We define the function $w(t)$:
    $$
    w(t) = y + \int_0^t u(s, \A y) ds
    $$
    Since $u(s, \A y)$ is continuous, $w(t)$ is continuously differentiable ($C^1$) with respect to $t$, and:
    $$ w'(t) = u(t, \A y) $$
    We can also pull the closed operator $\A$ outside the integral (since $u(s, \A y) \in D(\A)$):
    $$
    w'(t) = \A y + \int_0^t \frac{d}{ds} u(s, \A y) ds = \A \left[ y + \int_0^t u(s, \A y) ds \right] = Aw(t)
    $$
    At $t=0$, $w(0) = y$.
    
    This shows that $w(t)$ is a mild solution to $(CP)_{0,y}$. Because solutions are unique, it must coincide with our defined flow:
    $$ \forall t \ge 0, \quad w(t) = u(t,y) = T(t)y $$
    Differentiating both sides with respect to $t$ yields:
    $$ w'(t) = \frac{d}{dt} [u(t,y)] = \A u(t,y) = \A T(t)y $$
    However, from our earlier calculation, we also know $w'(t) = u(t, \A y) = T(t)\A y$. Equating the two expressions for $w'(t)$ gives the desired result:
    $$ T(t)\A y = \A T(t)y $$
\end{proof}

Now since $\rho(\A) \neq \emptyset$, then $\exists \lambda_0 \in \rho(\A), \forall y \in D(\A^2)$ set $x = (\lambda_0 I - \A)y$ then:
$$
T(t)x = T(t)(\lambda_0 I - \A)y = (\lambda_0 I - \A)T(t)y
$$
then
$$
\norm{T(t)x} \le |\lambda_0| \norm{T(t)y} + \norm{\A T(t)y} \le C_{\lambda_0} \norm{T(t)y} \le C_{\lambda_0} M e^{\omega t} |y|_G
$$
Where
\begin{align*}
	|y|_G = \norm{y} + \norm{\A y} &= \norm{y} + \norm{(\A -\lambda_0 I)y + \lambda_0 y} \\
				       &\le C_1 \norm{x} + \norm{x} + |\lambda_0| C_2 \norm{x} \\
				       &\le C_3 \norm{x}
\end{align*}
Finally we get:
$$
\norm{T(t)x} \le M^{'} e^{\omega t} \norm{x}
$$

Now by lemma \ref{app:sol:dense} and the fact that $\rho(\A) \neq \emptyset$ this bound can be extended to all $X$. And the operator commutes as well.

Finally, we need to show that $\A$ is the IG of $\left(T(t)\right)_{t \ge 0}$. Let $\A_1$ be the IG of $\left(T(t)\right)_{t \ge 0}$ then by definition of $T(t)$ we have $T(t)x = u(t, x)$ $\forall x \in D(\A)$ then:
$$
\frac{d}{dt} T(t)x = \A T(t)x \quad\quad \text{for } \; t \ge 0
$$
Which means in particular that $\frac{d}{dt} T(t)x\mid_{t = 0} = \A x$ then $\A_1 \supset \A$.

Let $\Re(\lambda) > \omega$ and let $y \in D(\A^2)$. Then:
$$
e^{-\lambda t} \A T(t) y = e^{-\lambda t} T(t) \A y = e^{-\lambda t} T(t) \A_1 y
$$
By integrating both sides from $0$ to $\infty$ we get:
$$
\A R(\lambda, \A_1) y = R(\lambda, \A_1) \A y = \A_1 R(\lambda, \A_1) y
$$
Since $R(\lambda, \A_1) \A_1 y = \A_1 R(\lambda, \A_1) y$ then $R(\lambda, \A_1) \A y = R(\lambda, \A_1) \A_1 y \; \forall y \in D(\A^2)$. Since $R(\lambda, \A_1) \A_1$ is uniformly bounded, $\A$ is closed and $\overline{D(\A^2)} = X$ we get $R(\lambda, \A_1) \A y = R(\lambda, \A_1) \A_1 y \; \forall y \in X$. This implies that $D(\A) \supset Rg(R(\lambda, \A_1)) = D(\A_1)$ hence $\A_1 \subset \A$ then $\A_1 = \A$ and the proof is complete.
\end{proof}

\end{document}

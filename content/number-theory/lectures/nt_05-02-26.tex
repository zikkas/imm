\documentclass[12pt, a4paper]{article}
\usepackage{amsmath, amssymb, amsthm}
\usepackage{bbm}
\usepackage{geometry}
\usepackage{hyperref}
\geometry{margin=1in}

\newtheorem{theorem}{Theorem}
\newtheorem{definition}{Definition}
\newtheorem{example}{Example}
\newtheorem{exercise}{Exercise}
\newtheorem{remark}{Remark}
\newtheorem{notation}{Notation}

\newcommand{\N}{\mathbb{N}}
\newcommand{\C}{\mathbb{C}}
\newcommand{\R}{\mathbb{R}}
\newcommand{\Z}{\mathbb{Z}}

\title{Analytical Number Theory: Lecture 01}
\date{5 Feb 2026}
\author{}

\begin{document}

\maketitle

\section{Arithmetic Functions}

\begin{definition}[Arithmetic Function]
An arithmetic function is a function $f: \N \rightarrow \C$.
\end{definition}

\begin{notation}
The set of all arithmetic functions is denoted by $\mathcal{F}$.
\end{notation}

\subsection{Examples}
The following are well-known arithmetic functions that play an important role in number theory:

\begin{enumerate}
    \item The Unit Function $u(n)$:
    \[
    u(n) = \begin{cases} 
    1 & \text{if } n=1 \\
    0 & \text{otherwise}
    \end{cases}
    \]
    \textit{Note:} the unit function can be written as follows:
    $$
    u(n) = \left\lfloor \frac{1}{n} \right\rfloor
    $$
    
    \item The Constant Function $\mathbbm{1}(n)$:
    \[
    \mathbbm{1}(n) = 1, \quad \forall n \ge 1
    \]
 
    \item The Euler Totient Function $\varphi(n)$:
    \[
    \varphi(n) = \#\{1 \le k \le n \mid \gcd(k, n) = 1\}, \quad \forall n \ge 1
    \]
    
    \item The Divisor Function $\sigma_{\alpha}(n)$:
    For any $\alpha \in \C$,
    \[
    \sigma_{\alpha}(n) = \sum_{d|n} d^{\alpha}
    \]
    
    \item The Divisor Count Function $\tau(n)$:
    This is a special case of the divisor function where $\alpha = 0$:
    \[
    \tau(n) = \sigma_0(n) = \sum_{d|n} 1
    \]
    Also denoted simply by $\sigma(n)$ when referring to the sum of divisors ($\alpha=1$).

    \item The Möbius Function $\mu(n)$:
    \[
    \mu(n) = \begin{cases} 
    1 & \text{if } n=1 \\
    0 & \text{if } p^2 | n \text{ for some prime } p \\
    (-1)^k & \text{if } n = p_1 p_2 \dots p_k \text{ (product of distinct primes)}
    \end{cases}
    \]
    
    \item The von Mangoldt Function $\Lambda(n)$:
    \[
    \Lambda(n) = \begin{cases} 
    \ln p & \text{if } n = p^k \text{ for some prime } p \text{ and } k \in \Z_{\ge 1} \\
    0 & \text{otherwise}
    \end{cases}
    \]
    
    \item The Omega Functions $\omega(n)$ and $\Omega(n)$:
    Let $n = p_1^{a_1} p_2^{a_2} \dots p_k^{a_k}$ be the prime factorization of $n$.
    \begin{itemize}
        \item $\omega(n) = k$ (counts the number of \textit{distinct} prime factors).
        \item $\Omega(n) = \sum_{i=1}^k a_i$ (counts the number of prime factors \textit{with multiplicity}).
    \end{itemize}
    
    \item The Liouville Function $\lambda(n)$:
    \[
    \lambda(n) = (-1)^{\Omega(n)}
    \]
\end{enumerate}

\section{Multiplicative Functions}

\begin{definition}[Multiplicative Function]
An arithmetic function $f$ is said to be multiplicative if it is not identically zero and:
\[
f(mn) = f(m)f(n) \quad \text{whenever } \gcd(m, n) = 1
\]
\end{definition}

\begin{notation}
The set of all multiplicative functions is denoted by $\mathcal{M}$.
\end{notation}

\begin{definition}[Completely Multiplicative Function]
An arithmetic function $f$ is called completely multiplicative if it is not identically zero and:
\[
f(mn) = f(m)f(n) \quad \text{for all } m, n \in \N
\]
\end{definition}

\begin{notation}
The set of all completely multiplicative functions is denoted by $\mathcal{CM}$. Clearly, $\mathcal{CM} \subset \mathcal{M} \subset \mathcal{F}$.
\end{notation}

\begin{theorem}
If $f$ is multiplicative, then $f(1) = 1$.
\end{theorem}

\begin{proof}
Since $f$ is not identically zero, there exists some $n \in \N$ such that $f(n) \ne 0$. Since $\gcd(n, 1) = 1$, we have:
$$
f(n) = f(n \cdot 1) = f(n)f(1)
$$
Hence, $f(n)(1 - f(1)) = 0$, and $f(n) \ne 0$ implying $f(1) = 1$.
\end{proof}

\subsection{Examples}
\begin{enumerate}
    \item The Unit Function $u(n)$ is completely multiplicative.

    \begin{proof}
    By definition, $u(1)=1$ and $u(n)=0$ for $n > 1$. Let $m, n \in \N$.
    \begin{itemize}
        \item If $m=1$ and $n=1$, then $mn=1$. We have $u(1) = 1$ and $u(1)u(1) = 1 \cdot 1 = 1$.
        \item If $m > 1$ or $n > 1$, then $mn > 1$, so $u(mn) = 0$. Since at least one of the inputs is greater than 1, either $u(m)=0$ or $u(n)=0$, making the product $u(m)u(n) = 0$.
    \end{itemize}
    Thus, $u(mn) = u(m)u(n)$ for all integers $m,n$.
    \end{proof}

    \item The Constant Function $\mathbbm{1}(n)$ is completely multiplicative.
    
    \begin{proof}
    For any $m, n \in \N$:
    \[
    \mathbbm{1}(mn) = 1 \quad \text{and} \quad \mathbbm{1}(m)\mathbbm{1}(n) = 1 \cdot 1 = 1
    \]
    The equality $\mathbbm{1}(mn) = \mathbbm{1}(m)\mathbbm{1}(n)$ holds for all integers.
    \end{proof}

    \item The Euler Totient Function $\varphi(n)$ is multiplicative.
\begin{proof} 
    We must show $\varphi(mn) = \varphi(m)\varphi(n)$ for $gcd(m,n)=1$.

    Using elementary number theory. Arrange the integers $1, 2, \dots, mn$ in a rectangular array with $n$ rows and $m$ columns:
    \[
    \begin{matrix}
    1 & 2 & \dots & r & \dots & m \\
    m+1 & m+2 & \dots & m+r & \dots & 2m \\
    \vdots & \vdots & & \vdots & & \vdots \\
    (n-1)m+1 & (n-1)m+2 & \dots & (n-1)m+r & \dots & nm
    \end{matrix}
    \]
    We need to count how many entries $x$ in this grid satisfy $gcd(x, mn) = 1$. Since $gcd(m,n)=1$, the condition $gcd(x, mn)=1$ is equivalent to satisfying both $gcd(x, m)=1$ and $gcd(x, n)=1$.

    In any column, all elements are congruent modulo $m$. Specifically, the element in the $r$-th column is of the form $km + r \equiv r \pmod m$.
    Therefore, $gcd(km+r, m) = gcd(r, m)$.
    An entry is coprime to $m$ if and only if it lies in a column $r$ where $gcd(r, m) = 1$. There are exactly $\varphi(m)$ such columns.

    On the other hand, fix one such valid column $r$ (where $gcd(r, m)=1$). The entries in this column are:
    \[
    r, \quad m+r, \quad 2m+r, \quad \dots, \quad (n-1)m+r
    \]
    This is an arithmetic progression with step $m$. Since $gcd(m, n)=1$, the elements $\{0, m, 2m, \dots, (n-1)m\}$ form a complete residue system modulo $n$. Shifting by $r$ implies that the sequence $r, m+r, \dots, (n-1)m+r$ is also a complete residue system modulo $n$.

    In a complete residue system modulo $n$, exactly $\varphi(n)$ numbers are coprime to $n$. Thus, in every valid column, there are exactly $\varphi(n)$ valid entries.

    Total valid numbers = (Number of valid columns) $\times$ (Valid entries per column)
    \[
    \varphi(mn) = \varphi(m) \times \varphi(n)
    \]
\end{proof}

\begin{proof}
    Another proof using the Chinese Remainder Theorem

    Let $R_{k}$ denote the ring of integers modulo $k$, and $R_{k}^{\times}$ denote the group of units in that ring. By definition, $\#R_{k}^{\times} = \varphi(k)$.

    Since $gcd(m,n)=1$, the Chinese Remainder Theorem states there is a ring isomorphism:
    \[
    \psi: \Z_{mn} \to \Z_{m} \times \Z_{n}
    \]
    defined by $\psi(x) = (x \pmod m, \, x \pmod n)$.

    An element $x \in \Z_{mn}$ is a unit (invertible) if and only if its image under $\psi$ is a unit in the product ring. An element $(a, b) \in \Z_{m} \times \Z_{n}$ is a unit if and only if $a$ is a unit in $\Z_{m}$ and $b$ is a unit in $\Z_{n}$.

    Therefore, the group of units of $\Z_{mn}$ is isomorphic to the direct product of the groups of units:
    \[
    \Z_{mn}^{\times} \cong \Z_{m}^{\times} \times \Z_{n}^{\times}
    \]
    Taking the order (size) of these groups:
    \[
    \#\Z_{mn}^{\times} = \#\Z_{m}^{\times} \cdot \#\Z_{n}^{\times}
    \]
    \[
    \varphi(mn) = \varphi(m)\varphi(n)
    \]
\end{proof}

    To show $\varphi(n)$ is not completely multiplicative, let $p$ be a prime. Then $\varphi(p^2) = p^2 - p$, but $\varphi(p)\varphi(p) = (p-1)^2$. These are not equal.

    \item The Divisor Functions $\sigma_{\alpha}(n) = \sum_{d|n} d^{\alpha}$ is multiplicative.

    \begin{proof}
    Let $gcd(m, n) = 1$. Since $m$ and $n$ are coprime, every divisor $d$ of the product $mn$ can be written uniquely as $d = ab$, where $a|m$ and $b|n$. Conversely, for any $a|m$ and $b|n$, the product $ab$ divides $mn$.
    \[
    \sigma_{\alpha}(mn) = \sum_{d|mn} d^{\alpha} = \sum_{a|m} \sum_{b|n} (ab)^{\alpha}
    \]
    Since $(ab)^{\alpha} = a^{\alpha}b^{\alpha}$, we can separate the sums:
    \[
    = \sum_{a|m} \sum_{b|n} a^{\alpha}b^{\alpha} = \left(\sum_{a|m} a^{\alpha}\right) \left(\sum_{b|n} b^{\alpha}\right) = \sigma_{\alpha}(m)\sigma_{\alpha}(n)
    \]

    To show it is not completely multiplicative, take $\alpha = 0$ then we have for any prime $p$ the following $\tau(p^2) = 3 \ne \tau(p)\tau(p) = 4$.
    \end{proof}

    \item The Möbius Function $\mu(n)$ is multiplicative.

    \begin{proof}
    Let $gcd(m, n) = 1$.
    \begin{itemize}
        \item If $m$ or $n$ is not square-free (divisible by some $p^2$), then $mn$ is also not square-free.
        \[
        \mu(mn) = 0 \quad \text{and} \quad \mu(m)\mu(n) = 0
        \]
        \item If both $m$ and $n$ are square-free, let $m = p_1 \dots p_k$ and $n = q_1 \dots q_j$. Since $gcd(m,n)=1$, all primes are distinct. $mn$ is the product of $k+j$ distinct primes.
        \[
        \mu(mn) = (-1)^{k+j} = (-1)^k (-1)^j = \mu(m)\mu(n)
        \]
    \end{itemize}
    To show $\mu(n)$ is not complete let $p$ be any prime, then $\mu(p^2) = 0$ while $\mu(p)^2 = (-1)^2 = 1$.
    \end{proof}

    \item The von Mangoldt Function $\Lambda(n)$ is not multiplicative function.

    \begin{proof}
    For a function to be multiplicative, it must satisfy $f(1)=1$.

    By definition, $\Lambda(1) = 0$. Hence, $\Lambda$ is not multiplicative.
    \end{proof}

    \item The Omega Functions $\omega(n)$ and $\Omega(n)$ are not multiplicative (Actually they are additive).

    \begin{proof}
    Both functions count prime factors. For $n=1$, the count is 0.
    \[
    \omega(1) = 0 \quad \text{and} \quad \Omega(1) = 0
    \]
    Since $f(1) \ne 1$, neither function is multiplicative.
    \end{proof}

    \item The Liouville Function $\lambda(n) = (-1)^{\Omega(n)}$ is completely multiplicative.

    \begin{proof}
    The function $\Omega(n)$ is completely additive, meaning:
    \[
    \Omega(mn) = \Omega(m) + \Omega(n) \quad \text{for all } m, n \in \N
    \]
    Now consider $\lambda(mn)$:
    \[
    \lambda(mn) = (-1)^{\Omega(mn)} = (-1)^{\Omega(m) + \Omega(n)} = (-1)^{\Omega(m)} \cdot (-1)^{\Omega(n)}
    \]
    \[
    \lambda(mn) = \lambda(m)\lambda(n)
    \]
    Since this relation holds for all integers $m, n$, the Liouville function is completely multiplicative.
    \end{proof}

\end{enumerate}

\begin{theorem}[Characterization of Multiplicative Functions]
Let $f$ be an arithmetic function such that $f(1) = 1$. Let $n = p_1^{a_1} \dots p_k^{a_k}$.
\begin{enumerate}
	\item \label{mf:char} $f$ is multiplicative if and only if:
    \[
    f(p_1^{a_1} \dots p_k^{a_k}) = f(p_1^{a_1}) \dots f(p_k^{a_k})
    \]
    \item \label{cmf:char} $f$ is completely multiplicative if and only if:
    \[
    f(p_1^{a_1} \dots p_k^{a_k}) = f(p_1)^{a_1} \dots f(p_k)^{a_k}
    \]
\end{enumerate}
\end{theorem}

\begin{proof}
Proof of $\implies$

Assume $f$ is multiplicative. We show that $f$ distributes over the prime factorization. Let $n = p_1^{a_1} p_2^{a_2} \dots p_k^{a_k}$. We proceed by induction on $k$, the number of distinct prime factors.

\begin{itemize}
    \item Base case ($k=1$): $f(p_1^{a_1}) = f(p_1^{a_1})$. This is trivial.
    \item Inductive step: Assume the formula holds for any integer with $k$ distinct prime factors.

    Let $n$ have $k+1$ distinct prime factors. We can write $n = m \cdot p_{k+1}^{a_{k+1}}$, where $m = p_1^{a_1} \dots p_k^{a_k}$.
    Since $p_{k+1}$ is distinct from primes in $m$, we have $gcd(m, p_{k+1}^{a_{k+1}}) = 1$.
    By the definition of multiplicativity:
    \[
    f(n) = f(m) f(p_{k+1}^{a_{k+1}})
    \]
    By the induction hypothesis, $f(m) = f(p_1^{a_1}) \dots f(p_k^{a_k})$.
    Substituting this back:
    \[
    f(n) = \left[ f(p_1^{a_1}) \dots f(p_k^{a_k}) \right] f(p_{k+1}^{a_{k+1}})
    \]
    Thus, the property holds for $k+1$ factors.
\end{itemize}

Proof by $\impliedby$

Assume that $f(p_1^{a_1} \dots p_k^{a_k}) = \prod_{i=1}^k f(p_i^{a_i})$ for any composite number. We must show that $f$ is multiplicative.
Let $m, n \in \N$ such that $gcd(m, n) = 1$.

Let the prime factorizations be:
\[
m = p_1^{a_1} \dots p_r^{a_r} \quad \text{and} \quad n = q_1^{b_1} \dots q_s^{b_s}
\]
Since $gcd(m, n) = 1$, the sets of primes $\{p_i\}$ and $\{q_j\}$ are disjoint (no prime appears in both sets). The prime factorization of the product $mn$ is simply the concatenation of these factors:
\[
mn = p_1^{a_1} \dots p_r^{a_r} q_1^{b_1} \dots q_s^{b_s}
\]
By the assumption (that $f$ breaks down into prime powers):
\[
f(mn) = f(p_1^{a_1}) \dots f(p_r^{a_r}) f(q_1^{b_1}) \dots f(q_s^{b_s})
\]
We can group these terms:
\[
f(mn) = \left[ \prod_{i=1}^r f(p_i^{a_i}) \right] \left[ \prod_{j=1}^s f(q_j^{b_j}) \right]
\]
Using the assumption again on $m$ and $n$ individually:
\[
f(m) = \prod_{i=1}^r f(p_i^{a_i}) \quad \text{and} \quad f(n) = \prod_{j=1}^s f(q_j^{b_j})
\]
Therefore:
\[
f(mn) = f(m)f(n)
\]
Since this holds for all coprime $m, n$, $f$ is multiplicative. This concludes the proof of part \ref{mf:char}.

Proof of part \ref{cmf:char} uses the same resoning.
\end{proof}

\begin{exercise}
Let $f(n) = [\sqrt{n}] - [\sqrt{n-1}]$. Show that $f$ is multiplicative.
\end{exercise}

\begin{exercise}
Let $f: \N \rightarrow \R$ be a non-decreasing multiplicative arithmetic function. For integers $a \ge 3$, let $R_t = \sum_{j=0}^t a^j$ and $S_t = a^t - \sum_{j=0}^{t-1} a^j$.

\begin{enumerate}
    \item Show that $f(S_t) \le (f(a))^t \le f(R_t)$.
    \item Deduce that for all integers $a, b, n > 2$, we have:
    \[
    f(b)^{r-1} \le f(n) \le f(a)^{r+2}
    \]
    where $r = [\log_a n]$ and similar definitions for bounds involving $b$.
    \item Show that $(f(a))^{\frac{1}{\log a}} = (f(b))^{\frac{1}{\log b}}$.
    \item Deduce that for all $n \ge 1$, $f(n) = n^k$ for some constant $k$.
\end{enumerate}
\end{exercise}

\section{Dirichlet Convolution}

\begin{definition}
Let $f$ and $g$ be two arithmetic functions. The Dirichlet convolution of $f$ and $g$, denoted $f * g$, is the arithmetic function defined by:
\[
(f * g)(n) = \sum_{d|n} f(d)g\left(\frac{n}{d}\right) = \sum_{ab=n} f(a)g(b), \quad \forall n \in \N
\]
\end{definition}

\begin{theorem}
The structure $(\mathcal{F}, +, *)$ is a commutative ring with unity and 
\[
	\mathcal{F}^{\times} = \left\{ f \in \mathcal{F} | f(1) \ne 0 \right\}
\]
\end{theorem}

\begin{proof}
We must verify the ring axioms. Let $f, g, h \in \mathcal{F}$.

First, let prove the commutativity of convolution ($f * g = g * f$)

By definition:
\[
(f * g)(n) = \sum_{d|n} f(d)g\left(\frac{n}{d}\right)
\]
Let $d' = n/d$. As $d$ runs through the divisors of $n$, $d'$ also runs through the divisors of $n$. We can substitute $d = n/d'$:
\[
= \sum_{n/d' | n} f\left(\frac{n}{d'}\right)g(d') = \sum_{d'|n} g(d')f\left(\frac{n}{d'}\right) = (g * f)(n)
\]

\bigskip

Second, we need to show the associativity of convolution ($(f * g) * h = f * (g * h)$)
Consider $((f * g) * h)(n)$:
\[
((f * g) * h)(n) = \sum_{d|n} (f * g)(d) h\left(\frac{n}{d}\right) = \sum_{d|n} \left[ \sum_{k|d} f(k)g\left(\frac{d}{k}\right) \right] h\left(\frac{n}{d}\right)
\]
Let $d = km$. Then $n/d = n/(km)$. The condition $d|n$ and $k|d$ is equivalent to $kmn = n$ for integers $k, m, r$ where $r = n/d$. Effectively, we sum over all triples $(a, b, c)$ such that $abc = n$:
\[
= \sum_{abc=n} f(a)g(b)h(c)
\]
Similarly, expanding $f * (g * h)$:
\[
(f * (g * h))(n) = \sum_{a|n} f(a) (g * h)\left(\frac{n}{a}\right) = \sum_{a|n} f(a) \sum_{b c = n/a} g(b)h(c) = \sum_{abc=n} f(a)g(b)h(c)
\]
Since both expressions equal $\sum_{abc=n} f(a)g(b)h(c)$, the operation is associative.

\bigskip

Now we need to show the existence of Unity.

We claim $u(n)$ is the identity.
\[
	(f * u)(n) = \sum_{d|n} f(d)u\left(\frac{n}{d}\right) = f(n)u(1) + \sum_{\substack{d|n \\ d < n}} f(d)u\left(\frac{n}{d}\right) = f(n)
\]
The term $u(n/d)$ is 0 since $n/d > 1$. Thus $f * u = f$ and by commutativity $u * f = f$.

\bigskip

Then we show that the convultion distribute over addition
\[
(f * (g + h))(n) = \sum_{d|n} f(d)(g + h)\left(\frac{n}{d}\right) = \sum_{d|n} \left( f(d)g\left(\frac{n}{d}\right) + f(d)h\left(\frac{n}{d}\right) \right)
\]
\[
= \sum_{d|n} f(d)g\left(\frac{n}{d}\right) + \sum_{d|n} f(d)h\left(\frac{n}{d}\right) = (f * g)(n) + (f * h)(n)
\]

Hence $(\mathcal{F}, +, *)$ is a commutative ring with unity.

\bigskip

Now to prove $\mathcal{F}^{\times} = \left\{ f \in \mathcal{F} | f(1) \ne 0 \right\}$.

Assume $f$ is invertible. Then there exists $g$ such that $(f * g)(n) = u(n)$ for all $n \in \N$.
Evaluating the convolution at $n=1$:
\[
(f * g)(1) = \sum_{d|1} f(d)g\left(\frac{1}{d}\right) = f(1)g(1)
\]
By the definition of the unit function, $u(1) = 1$. Therefore:
\[
f(1)g(1) = 1.
\]
Since the product of two complex numbers is non-zero, neither factor can be zero. Hence $f(1) \neq 0$.

\bigskip

On the other hand, assume $f(1) \neq 0$. We wish to construct an arithmetic function $g$ such that $(f * g)(n) = u(n)$ for all $n$.
We define $g(n)$ inductively on $n$.

Base case ($n=1$):
We require $(f * g)(1) = f(1)g(1) = u(1) = 1$.
Since $f(1) \neq 0$, we can uniquely define:
\[
g(1) = \frac{1}{f(1)}.
\]

Inductive step ($n > 1$):
Assume that the values $g(k)$ have been uniquely determined for all $k < n$. We examine the condition for $n$:
\[
(f * g)(n) = \sum_{d|n} f(d)g\left(\frac{n}{d}\right) = u(n) = 0 \quad (\text{since } n > 1)
\]
We isolate the term where $d=1$ in the sum:
\[
f(1)g(n) + \sum_{\substack{d|n \\ d > 1}} f(d)g\left(\frac{n}{d}\right) = 0
\]
Rearranging the equation to solve for $g(n)$:
\[
f(1)g(n) = - \sum_{\substack{d|n \\ d > 1}} f(d)g\left(\frac{n}{d}\right)
\]
Since $f(1) \neq 0$, we can divide by it:
\[
g(n) = \frac{-1}{f(1)} \sum_{\substack{d|n \\ d > 1}} f(d)g\left(\frac{n}{d}\right)
\]
Notice that for any divisor $d > 1$, the argument $\frac{n}{d}$ is strictly less than $n$. Thus, the value $g(\frac{n}{d})$ is already known by the induction hypothesis. This formula uniquely determines $g(n)$.

Hence $\mathcal{F}^{\times} = \left\{ f \in \mathcal{F} | f(1) \ne 0 \right\}$.
\end{proof}

\begin{theorem}
The set of multiplicative functions $\mathcal{M}$ is a subgroup of the group of units $\mathcal{F}^{\times}$. That is:
\begin{enumerate}
    \item If $f$ and $g$ are multiplicative, then $f * g$ is multiplicative.
    \item If $f$ is multiplicative, then $f^{-1}$ is multiplicative.
\end{enumerate}
\end{theorem}

\begin{proof}
Let $f$ and $g$ be multiplicative functions. Let $h = f * g$.
We must show that $h(mn) = h(m)h(n)$ whenever $gcd(m, n) = 1$.

Let $m, n \in \N$ such that $gcd(m, n) = 1$.
\[
h(mn) = \sum_{d|mn} f(d)g\left(\frac{mn}{d}\right)
\]
Since $gcd(m, n) = 1$, every divisor $d$ of $mn$ can be written uniquely as $d = ab$, where $a|m$ and $b|n$. Furthermore, $gcd(a, b) = 1$ and $gcd(\frac{m}{a}, \frac{n}{b}) = 1$.
Substituting $d=ab$ into the sum:
\[
h(mn) = \sum_{a|m} \sum_{b|n} f(ab)g\left(\frac{mn}{ab}\right)
\]
Using the multiplicativity of $f$ and $g$:
\[
f(ab) = f(a)f(b) \quad \text{and} \quad g\left(\frac{m}{a} \cdot \frac{n}{b}\right) = g\left(\frac{m}{a}\right)g\left(\frac{n}{b}\right)
\]
Substitute these into the summation:
\[
h(mn) = \sum_{a|m} \sum_{b|n} f(a)f(b) g\left(\frac{m}{a}\right)g\left(\frac{n}{b}\right)
\]
We can factor the double sum into the product of two single sums:
\[
h(mn) = \left( \sum_{a|m} f(a)g\left(\frac{m}{a}\right) \right) \left( \sum_{b|n} f(b)g\left(\frac{n}{b}\right) \right)
\]
By definition of convolution, these factors are exactly $(f * g)(m)$ and $(f * g)(n)$.
\[
h(mn) = h(m)h(n)
\]
Thus, $f * g$ is multiplicative.

\bigskip

Let $f$ be a multiplicative function. Since $f(1) = 1 \ne 0$, $f$ has an inverse $g = f^{-1}$. We must show that $g$ is multiplicative.

We proceed by induction on the product $mn$.
We want to show $g(mn) = g(m)g(n)$ for all $gcd(m,n)=1$.

Base case ($mn=1$):
Since $m=1, n=1$, we have $g(1) = 1/f(1) = 1$. Thus $g(1) = g(1)g(1)$ holds.

For the inductive step, assume $g(ab) = g(a)g(b)$ for all coprime $a, b$ with $ab < mn$.
By definition of the inverse, $(f * g)(mn) = u(mn)$. Since $mn > 1$, $u(mn) = 0$.
\[
\sum_{d|mn} f(d)g\left(\frac{mn}{d}\right) = 0
\]
Since $gcd(m,n)=1$, every divisor $d$ of $mn$ is uniquely $d=ab$ where $a|m, b|n$.
\[
\sum_{a|m} \sum_{b|n} f(ab)g\left(\frac{mn}{ab}\right) = 0
\]
Using the multiplicativity of $f$ and separating the term where $a=1, b=1$ (so $d=1$):
\[
g(mn) + \sum_{\substack{a|m, b|n \\ ab > 1}} f(a)f(b) g\left(\frac{mn}{ab}\right) = 0
\]
For $ab > 1$, we have $\frac{mn}{ab} < mn$. Since $gcd(\frac{m}{a}, \frac{n}{b})=1$, the induction hypothesis applies:
\[
g\left(\frac{mn}{ab}\right) = g\left(\frac{m}{a}\right)g\left(\frac{n}{b}\right)
\]
Substitute this back:
\[
g(mn) = - \sum_{\substack{a|m,\, b|n \\ ab > 1}} f(a)g\left(\frac{m}{a}\right) f(b)g\left(\frac{n}{b}\right)
\]
We recognize the sum on the right as the expansion of the product $(f*g)(m) \cdot (f*g)(n)$, excluding the term for $a=1, b=1$:
\[
(f*g)(m) \cdot (f*g)(n) = \left(\sum_{a|m} f(a)g\left(\frac{m}{a}\right)\right) \left(\sum_{b|n} f(b)g\left(\frac{n}{b}\right)\right)
\]
Since $mn > 1$ and $gcd(m,n)=1$, at least one of $m,n > 1$. Thus, $(f*g)(m)(f*g)(n) = u(m)u(n) = 0$.
The full summation over all $a,b$ is 0. The sum restricted to $ab > 1$ is simply the total sum minus the $a=b=1$ term:
\[
\sum_{\substack{a|m,\, b|n \\ ab > 1}} f(a)g\left(\frac{m}{a}\right) f(b)g\left(\frac{n}{b}\right) = 0 - [f(1)g(m) \cdot f(1)g(n)] = -g(m)g(n)
\]
Then
\[
g(mn) = - (-g(m)g(n)) = g(m)g(n)
\]\end{proof}

\begin{remark}
	The following results can be concluded:
\begin{enumerate}
	\item The convolution of two completely multiplicative functions is not necessarily completely multiplicative (though it is multiplicative). For example, $\mathbbm{1} * \mathbbm{1} = \tau$. While $\mathbbm{1}$ is completely multiplicative, $\tau$ is not.
	\item If $f$ and $g$ are arithmetic functions and $f * g$ is a multiplicative functions, it does not mean either $f$ or $g$ is multiplicative.
	\item To prove a function $f$ is multiplicative, sometimes it may be easier to find two multiplicative functions ($g$ and $h$) such that $g * h = f$, then $f$ is multiplicative.
\end{enumerate}
\end{remark}

\begin{exercise}
	Prove the following identity:
	\[
		\frac{n}{\varphi(n)} = \sum_{d|n}{\frac{\mu^2(d)}{\varphi(d)}}
	\]
\end{exercise}


\begin{exercise}
	Give a simple expression of the following sums:
	\begin{enumerate}
		\item $\sum_{d|n}{\mu(d)\tau(\frac{n}{d})}$
		\item $\sum_{d|n}{\mu^2(d)\varphi(d)}$
		\item $\sum_{d|n}{\mu(d)\tau(d)}$
	\end{enumerate}
\end{exercise}

\end{document}
